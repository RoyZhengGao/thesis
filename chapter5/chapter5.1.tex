\section{Introduction}

Prior investigations on review summarization, mainly follow Natural Language Generation (NLG) approaches, such as \cite{sutskever2011generating} uses new Recurrent Neural Network (RNN) variant that uses gated connections to construct a character-level text generation mode and \cite{li2017neural} designs a multi-task model to predict rating and generate review summarization simultaneously using pairwise user-product relationship.  \cite{wang2017opinion} uses a memory network for review summarization generation.

However, all these models are originally designed for static review summarization and take product reviews as model input. They are vulnerable to depict product characteristic changes because of (real-time review) data sparsity. For instance, after investigating 2.16 billion products sold by \textit{Taobao}, a world-leading online shopping website owned by Alibaba, only 0.05\% of products are able to gather more than 100 reviews within a three-day window. Thus, review-based approaches are not feasible for dynamic summarizations in large scope because of the lack of instant reviews. 

On the other hand, user behavior offers an alternative to address characteristic dynamics. Based on the statistics of the \textit{Taobao} collection, more than 2.53\% of products can receive more than 100 multi-type user behaviors (e.g., \textit{`Click'} or \textit{`Purchase'}) within a three-day window where the coverage is 50 times greater than the review scope. Rational Choice Theory \cite{blume2008rationality}, on the theory side, proves that user shopping behavior rationality has a coherent relationship with the product peculiarity. 

Motivated by all aforementioned scenarios, I propose a  \textbf{B}ehavior based \textbf{D}ynamic \textbf{S}ummarization (BDS) model to accommodate user behavior for dynamic product summarization with the help of community detection techniques. The user shopping preference is stable in a relatively long-term period \cite{brouwer2017choice}, which offers us the theoretical feasibility to learn product behavior representation from user dynamic behavior and consistent shopping preference. The learned representation supports neighbor product selection from a group of seed products with abundant instant reviews (\textbf{Task 1}) and meanwhile implicitly helps to generate summarization from product own descriptive phrases and neighbor products' filtered sentimental phrases (\textbf{Task 2}). As both user behavior and seed products' instant reviews are changed across time, the generated summarizations and neighbor products from detected communities are associated with changes as well. 


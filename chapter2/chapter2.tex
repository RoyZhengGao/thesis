\chapter{Related Works}  \label{ch:review}

In this chapter, I select the most outstanding studies based on a self-defined criteria (either published in a set of pre-selected venues or performed the highest impact by receiving at least fifty citations). To better introduce these papers in a well-organized manner, I categorize them into five following tracks borrowing ideas from aforementioned survey studies \cite{fortunato2010community, fortunato2016community, coscia2011classification}. In detail, papers in the Graph Type track focus on detecting communities in different types of graphs, such as heterogeneous or sparse graphs. In the Task track, the selected studies aim to solve particular tasks in community detection, such as deciding the correct number of communities. In the Methodological track, the introduced studies solve the general community detection problem via different types of model frameworks such as Modularity or spectral methods. In the Application track, selected studies discuss community detection applications and how to apply them to other disciplines. In the last Evaluation track, it lists papers to summarize the evaluation metrics widely used for model justification and comparison. 


\section{Graph Types}
\subsection{Heterogeneous \& Multi-layer Graph}
hetero: \cite{huang2018overlapping},\cite{sun2013pathselclus},\cite{gupta2013community},\cite{he2015stochastic},\cite{sun2012relation}incomplete graph,
\cite{radicchi2013detectability},\cite{zhou2013social},
multi-layer: \cite{kim2015community} survey,\cite{bazzi2016community},\cite{de2015identifying},\cite{valles2016multilayer},\cite{huang2018harmonic}

\subsection{Sparse Graph}
\cite{krzakala2013spectral},\cite{chen2012clustering},\cite{amini2013pseudo},\cite{banks2016information},\cite{chin2015stochastic},\cite{guedon2016community},\cite{mirshahvalad2012significant},

\subsection{Dynamic Graph}
\cite{peixoto2017modelling},\cite{kim2013nonparametric},\cite{delvenne2010stability},\cite{ghasemian2016detectability},\cite{xu2015stochastic},\cite{sarzynska2016null},\cite{gauvin2014detecting},\cite{xu2014dynamic},\cite{huang2014querying},\cite{tang2011identifying},\cite{hartmann2016clustering},\cite{chen2013detecting},\cite{wang2013neiwalk},\cite{wang2013dynamic},\cite{folino2013evolutionary},

\subsection{Large Graph}
\cite{harenberg2014community}(survey),\cite{li2015uncovering},\cite{jeub2015think},\cite{de2014mixing},\cite{macropol2010scalable},\cite{kollios2013clustering},\cite{wang2011detecting},\cite{hallac2015network},\cite{bhatia2018dfuzzy},\cite{peixoto2015model},\cite{tsourakakis2017scalable},\cite{li2013efficient},\cite{prat2014high},\cite{whang2012scalable},\cite{satuluri2011local},\cite{spielman2013local},\cite{liu2013large}

\subsection{Attribute Graph}
\cite{zhang2019attributed},\cite{jin2019graph},\cite{yang2013community},\cite{he2017joint},\cite{wang2016semantic},\cite{qi2012community},\cite{huang2015dense},\cite{han2015probabilistic},\cite{xu2012model},\cite{pool2014description},\cite{zhou2010clustering}

\subsection{Summary}
\section{Tasks}
In this section, research works about five  popular community detection tasks are summarized. The five selected tasks are independent with each other and tackle community detection problems from various perspectives. 

\subsection{Overlapping Community Detection}
\begin{figure}
	% \setlength{\belowcaptionskip}{-10pt}
	\center
	\includegraphics[width=\columnwidth]{img/chapter2/overlapping.pdf} 
	\caption{Overlapping Community Structure where nodes can belong to multiple communities. The figure is contributed from \cite{yang2013overlapping}.}
	\label{fig:c2_overlapping}
\end{figure}  


Overlapping community detection is one of the most fundamental topics and many relevant studies are published. By definition, a node $v$ can belong to multiple communities simultaneously so as to cause the overlaps between communities. Figure \ref{fig:c2_overlapping} visualizes the community structure under overlapping circumstance. The nodes marked in color yellow are affiliated with both community $A$ and $B$. Table \ref{tab:c2_overlapping} categorizes and summarizes the focus of each mentioned study for overlapping community detection.

\begin{table}
	% \scriptsize
	\centering
	%   \vspace{-3em} 
	% \renewcommand{\tabcolsep}{2pt}
	\begin{tabular}{|p{5cm}|p{9cm}|} \hline
		\textbf{References} &  \textbf{Main Idea} \\ \hline
		\cite{coscia2012demon,whang2013overlapping,whang2016overlapping,huang2018overlapping,wang2017overlapping} & Local search and seed expansion\\ \hline
		\cite{yang2013overlapping,zhang2015incorporating,zhang2016modeling,eustace2015overlapping,jin2015combined}& Nonnegative matrix factorization\\ \hline
		\cite{gopalan2013efficient,jin2016detect,}& Bayesian, generative model\\ \hline
		\cite{yang2012community}& Affiliation network\\ \hline
	\end{tabular}
	\caption{The main ideas of mainly introduced overlapping community detection approaches.}
	\label{tab:c2_overlapping}
	
\end{table} 

\cite{amelio2014overlapping} is a survey paper published in 2014. Although the methods introduced within it are sort of dated, it still offers covers several main track methodologies (node seed and local expansion, clique expansion, and label propagation) and particular scenarios (link clustering and dynamic graphs). It draws a conclusion that there is no universal method to deal with all types of graphs with different characteristics in sparsity, degree distribution, and overlap percentage among communities. In the end, it also points out two essential questions to be solved for later researchers, which is ``when to apply overlapping methods and how significant the overlapping is''.  \cite{xie2013overlapping} is another authentic survey paper which reviews a set of methods, benchmarks and evaluation metrics. Besides that, it also reviews papers using stochastic block model or density based models. To study the performance differences among models,  it introduces several benchmark graphs in which node ground truth community is known, i.e. LFR benchmark graph\footnote{https://sites.google.com/site/andrealancichinetti/files}. A set of evaluation metrics including Normalized Mutual Information (NMI) and Omega index are introduced as well. Meanwhile, empirical studies are applied and evaluated on all aforementioned methods using different benchmark graphs. In the end, the paper discovers the sensitivities of different models in sparse graphs appeared in real-world social networks.

DEMON model \cite{coscia2012demon} unveils its overlapping communities with a local-first approach by grouping ego neighbor nodes into the same clusters and finally merges the local communities into a global collection. In the local-first grouping process, DEMON applies a EgoMinusEgo function to first extract ego-based subgraphs for each node $v$ where the node set is node $v$ and all its neighbor nodes $\mathcal{N}(v)$  and the edge set is all graph edges between the selected nodes ($v$ and $\mathcal{N}(v)$). After that, each ego-based subgraph will remove the node $v$ and all its associated edges to achieve an EgoMinusEgo subgraph. An label propagation community detection method is subsequently applied on each EgoMinusEgo subgraph and select a set of largest clusters to best cover the entire graph. In the end, a merging process is applied on the previously generated clusters according to their node similarities and construct the final community partition result. LOSP model \cite{he2015detecting} is another method to explore local neighborhood structures of each node. It defines a Local Spectral Subspace using the first $d$ eigenvectors from the normalized adjacency matrix. In each potential community $c$, a set of seed nodes are given, and an iterative process is applied with the help of seed nodes and the Local Spectral Subspace to rank the top $N_c$ nodes with highest random walk probabilities to appear in the current community. Those nodes are finally regarded as other latent members in each community $c$. 

LOSP model offers a great insight to use seed node expansion to detect overlapped communities. Inspired by LOSP model, \cite{whang2013overlapping,whang2016overlapping} propose a four-phase model including filtering, seeding, seed set expansion, and propagation. The natural of the model first filters out the regions of the graph which don't involve overlapping structure. A seeding strategy inspired by Kmeans selects a set of seed nodes with small Conductance relationship. A seed set expansion approach is further applied by taking advantage of personalized PageRank model to construct raw communities near each seed node. In the last propagation step, the raw communities are expanded again to the regions previously removed in the filtering phase. Similarly, \cite{yang2017finding} also designs a new seeding strategy and expands a set of nodes from seed nodes via a personalized PageRank model. Thereafter it develops a model on the expanded nodes to detect overlapped communities. The seed nodes are the nodes in the maximum spanning tree of the original graph. Personalized PageRank model helps to expand the seed nodes and merge them into raw community structure by maximizing modularity score of the graph. The last optimization step is to merge communities when the shared node ratio of two communities is above an overlapping threshold.

OCD-HSN model \cite{huang2018overlapping} contains a seed selection step as well as community initialization and expansion step to group nodes into clusters in an efficient manner. It claims to be the first research work for overlapping community detection in heterogeneous graph containing both undirected and directed edges. Based on user semantic interests and social connections, this study constructs a heterogeneous graph to hold all types of user profiles. A set of seed nodes are selected based on their centrality and conductance in the graph. The neighbor nodes of the seed nodes naturally  construct overlapped communities. Later on, a fitness evaluation process is leveraged to add or remove nodes  based on the node connectivity in each community. 

Structural centers are defined and utilized in \cite{wang2017overlapping} to support local expansion for overlapping community detection, which are the nodes that have high node degrees and are also far away from other structural centers.


By natural, nonnegative matrix factorization can solve overlapping community detection problem as it can directly learn low dimensional node representation matrix from the graph matrices such as adjacency matrix. Each column in the reduced node represention matrix represents an eigenvector, which also can be regarded as a community. Therefore, the learned node representation can be regarded as the node affiliation distribution over communities. BIGCLAM model \cite{yang2013overlapping} is a classic overlapping community detection model which can take care of large-scale graphs. In its model, $F$ represents a nonnegative matrix where $F_{uc}$ is the affiliation score of node $u \in V$ in community $c \in C$. Given an unlabeled and undirected network $G(V, E)$, BIGCLAM aims to re-generate all graph edges using node's community affiliation distribution. It aims to maximize the probability of node $u$ and $v$ if there is an edge between them and minimize the probability if there is no edge.  Therefore, the objective function $l(F)$ to maximize turns to be:

\begin{equation}
l(F) = \sum_{(u,v) \in E} log(1-exp(-F_uF_v^T)) - \sum_{(u,v) \notin E} F_uF_v^T
\end{equation}

The number of communities can also be determined in the model through an empirical test.
 
PNMF model \cite{zhang2015incorporating} is also a nonnegative matrix factorization model.  Unlike conventional matrix factorization models to directly decompose the original adjacency matrix, PNMF considers the situation that a node prefers its neighbor nodes. Therefore, instead of re-generating the edges between two nodes, this paper considers node triplet $(i,j,k)$.  $j>_{i} k$ denotes that $ j \in \mathcal{N}(i)$ and $k \notin \mathcal{N}(i)$ so that node $i$ prefers $j$ to $k$. And the overall goal of this approach is to maximize this preference likelihood that neighbor nodes should be preferred than non-neighbor nodes:
\begin{equation}
\begin{aligned}
l(F) &= \sum_{(i,j,k) \in S} log(p(j>_{i}k|F) )-\lambda \cdot reg(F) \\
 &=  \sum_{(i,j,k) \in S} log(\sigma(F_iF_j^{T} - F_iF_k^{T}))-\lambda \cdot reg(F)
\end{aligned}
\end{equation}
 where $S$ is pre-sampled node triplet collection which satisfies the node preferences. $reg(F)$ is the regularization term added to avoid overfitting and $\lambda$ is the regularization weight.
 
HNMF \cite{zhang2016modeling} is another nonnegative matrix factorization model which considers both-sided relationships between edges and communities. From the community-to-edge perspective, it takes advantages of PNMF model to maximize the log likelihood of node connection preferences in all sampled node triplets. From the edge-to-community perspective, it uses negative sampling strategy in a skip-gram model to maximize the probability if two nodes are neighbor nodes and minimize the probability if they are not. The final loss is the loss sum from both PNMF and the negative sampling, which is aimed to be maximized in the joint training process. 
 
NRATIO model \cite{eustace2015overlapping} generates a vertex neighborhood ratio matrix to substitute original adjacency matrix or Laplacian matrix. This matrix refines the graph adjacency matrix where two nodes are connected only if their common neighbor nodes surpasses the average number of neighbors in all pairwise nodes. In the next step, nonnegative matrix factorization is applied on the refined matrix to learn a community distribution over each node. As the vertex neighborhood ratio matrix reduces the influence of unrelated nodes in community structures, the number of data points in the matrix are significantly reduced, which fastens the running speed.

\cite{jin2015combined} first describes a stochastic block model to accommodate both node and edge communities, and then uses conductance measurement to fine-tune the learned node community distribution. One outstanding point of this paper is that the model considers edge communities. By definition, in the adjacency matrix $A$, $a_{ij} = 1$ if node $v_i$ and node $v_j$ are connected through an edge. And, in bipartite graph matrix $B$, $b_{ij} = 1$ if node $v_i$ and edge $e_j$ are directly linked each other. The paper aims to learn a node community affiliation matrix $H$ where $h_{ij}$ denotes the propensity of node $v_i$ belonging to community $c_j$ and an edge community affiliation matrix $W$ where $w_{ij}$ denotes the propensity of edge $e_i$ belonging to community $c_j$. In the end, to use the node/edge community information ($W$ and $H$) to re-construct the two matrices $A$ and $B$, the objective function is defined as follows:

\begin{equation} 
	\mathcal{O}(H,W) = ||A-HH^T||^2_F + \lambda ||B-WH^T||^2_F
\end{equation}
where $||\cdot||_F$ is the Frobenius norm, $H$ and $W$ have to be nonnegative matrices.

AGM model \cite{yang2012community} is a preliminary study using affiliation network to re-generate the original social network. Given only the overlapping community affiliation of each node, it learns the  reproduction of original links between nodes in order to construct the social network. The edge generation probability is defined as: 
\begin{equation}
p(u,v) = 1-\prod_{c \in C_{uv}} (1-p_c)
\end{equation}
where $C_{uv}$ are a set of shared communities for node $v$ and $u$. $p_k$ refers to the probability of an edge forming between two nodes in the community $c$. 

\cite{gopalan2013efficient} uses a mixture membership stochastic block model to learn node overlapping communities. The overall framework is a Bayesian approach. It assumes the overlapped community memberships of each node $v_i$ is associated with a Dirichlet distribution $\theta_i$. For each pair of node $v_i$ and $v_j$, it draws a community indicator $z_{i \rightarrow j}$ from node $v_i$’s community membership $\theta_i$ and then draws  a community indicator  $z_{i \leftarrow j}$  from node $v_i$’s community membership $\theta_j$. Each indicator points to one
of the $|C|$ communities. Finally, it draws an edge between two nodes with the generation probability:
\begin{equation} 
p(y_{ij} = 1|z_{i \rightarrow j},z_{i \leftarrow j}) =
\begin{cases}
z_{i \rightarrow j},       & \quad  z_{i \rightarrow j} = z_{i \leftarrow j}\\ 
\epsilon,  & z_{i \rightarrow j} \neq z_{i \leftarrow j}\\ 
\end{cases}
\end{equation}
The whole process is optimized and all nodes' $\theta$ are learned by maximizing the edge generation probability $p(y_{ij} = 1)$from the actual graphs.  $\epsilon$ is a small constant. The
parameter $\beta_{z_{i \rightarrow j}}$ is the density of community $z_{i \rightarrow j}$.

Similarly, \cite{jin2016detect} also introduces a Bayesian based generative model. It assumes a Poisson distribution for node community distribution and is optimized using a Expectation-Maximization (EM) approach. Node communities are selected according to its learned community distribution and related community conductance in the end.

\subsection{Number of Communities}
Among all sorts of community detection models, only few of them can automatically determine the number of communities given the model nature. Most of them need to specify the community number in advance before applying their main processes. Therefore it is still an open question about how to choose the exact community number. Many in-depth researches draw their conclusions either from empirical studies or mathematical proofs. In this section, several works particularly exploring this research question are introduced and discussed, which offers general insights for potential community number selection.

\cite{newman2016estimating} proposes a statistical model using Bayesian inference analysis. It finds the correct number of communities by maximizing the integrated likelihood of graph structure and observed community partition using a generative model. In detail, given the prior knowledge of graph structure (adjacency matrix $A$) and community partition $C$, it uses Markov chain Monte Carlo (MCMC) importance sampling to calculate the posterior distribution of community number $|C|$.

\cite{le2015estimating} develops an efficient model to study community numbers based on graph spectral properties. It statistically proves the number of positive and most informative eigenvalues as the estimated number of communities on five spectral clustering methods which use either the non-backtracking matrix or the Bethe Hessian matrix. 

Moreover, two motivations happened in stochastic block models urge the exploration of community numbers. First, edges are not necessarily independent if only the communities of their endpoint nodes are given. Second, the loss of precision occurred in spectral clustering is inevitable. Therefore, through a set of rigorous proofs, \cite{saldana2017many} proposes a composite likelihood BIC (CL-BIC) model to handle the community number selection  happened in stochastic block model (SBM), degree-corrected block model (DCBM) and mixture-membership block model (MMB). 

For other works relevant to community number selection, \cite{kawamoto2017cross,chen2018network} both leverage the leave-one-out cross-validation to determine the community number by optimizing the edge prediction error and Bethe free energy in stochastic block model.  \cite{kawamoto2018comparative} conducts a comparative analysis on various community detection models (stochastic block model, greedy methods, statistical inference models and spectral methods) to track the changes of assessment criteria (modularity, map equation, Bethe free energy and prediction error and isolated eigenvalues) associated with various of community numbers.

\subsection{Community Search}
Instead of generating communities for all nodes in the graph, community search based approaches aim to find the top relevant subgraphs or communities that match particular purposes, which can be regarded as a mixture of community detection and information retrieval. Currently, there are two main types of search scenarios. One is that given a set of query nodes, return the densest subgraph containing query nodes. The other is given a set of query nodes, return their top $k$ most relevant communities. Relevant works are summarized by main ideas in Table \ref{tab:c2_search}.

\begin{table}
	% \scriptsize
	\centering
	%   \vspace{-3em} 
	% \renewcommand{\tabcolsep}{2pt}
	\begin{tabular}{|p{4.5cm}|p{9.5cm}|} \hline
		\textbf{References} &  \textbf{Main Idea} \\ \hline
		\cite{sozio2010community,cui2014local,barbieri2015efficient,wu2015robust,huang2015approximate} & Return the densest subgraph containing all query nodes\\ \hline
		\cite{chen2012dense,qin2015locally,lancichinetti2011finding,huang2014querying,zheng2017finding,yan2019constrained,li2015influential}& Return top $k$ most relevant/influential communities with or without query nodes \\ \hline 
	\end{tabular}
	\caption{The main ideas of community search approaches.}
	\label{tab:c2_search}
	
\end{table} 


\cite{sozio2010community} is a very classic paper of community search, which is defined as given a graph $G(V,E)$ and a set of query nodes, seeking a a subgraph of $G$ which contains all the query nodes and meanwhile is densely connected. The nodes with trivial degrees and far away from query nodes are filtered out from the community. The final goal of this paper is to maximize the minimum degree of a subgraph containing all query nodes. A greedy solution is proposed by deleting each node with minimum degree at each step. The iterative process terminates either at least one of the query node reaches to the minimum degree or is no longer connected with the rest of nodes. Further two heuristic approaches are extended to detect the community with an upper bound size restriction. 

\cite{cui2014local} is a subsequent work of the previous one, which aims to detect the best community for a given node. It formulates and solves two community search problems: communitysearch with a threshold constraint (CST) problem and community search with a maximality constraint (CSM) problem. Two approaches including global search and local search are proposed where the second approach is more efficient than the first one because of reducing searching space significantly.  

\cite{barbieri2015efficient} is another follow-up work of \cite{sozio2010community}. It improves the model efficiency and accuracy as one limitation of the previous work  is that it tends to find quite large solutions, which negatively affects the detected community accuracy. Unlike the \cite{sozio2010community} in which the community size is constrained, it aims to detect the smallest-size community among all optimal ones. In detail, it computes the core decomposition and organizes the retrieved k-cores (maximal subgraphs where each nodes are connected at least $k$ other nodes in a subgraph). In the end, a greedy-connection approach is leveraged to use a local search method to reduce the potential subgraph size (greedy step) and a Steiner Tree-based strategy to find the minimum number of nodes that make all query nodes connected.
 
Given a query node, \cite{wu2015robust} formulates a query biased densest connected subgraph (QDC) problem which aims to find the densest subgraphs that contains the query nodes and is also internally connected. A densest subgraph can be regarded as a local community in this paper. To reduce a free rider effect which involves irrelevant nodes into densest subgraphs, a random walk based weighting scheme is proposed to set higher penalties to these irrelevant nodes when they are included in the subgraphs/communities. 

\cite{huang2015approximate} solves a closest truss community (CTC) problem which finds a connected k-truss subgraph with the largest $k$ and minimum graph diameter.  By its definition, a k-truss subgraph requires each edge is contained in at least $k-2$ triangles  in the subgraph. A greedy algorithm is proposed to first find a raw CTC that contains all query nodes, then iteratively removes the furthest nodes in the CTC to finally achieve the optimized subgraph.
 
\cite{chen2012dense} introduces a partial clustering model to extract the most dense subgraphs, meanwhile it does not require to determine the number of subgraphs in advance. The model is inspired by the matrix blocking problem to reorder the adjacency matrix $A$ and to find the dense diagonal blocks as extracted subgraphs. It considers three types of graph scenarios including undirected graph, directed graph and bipartite graph. By calculating the pairwise columns in the adjacency matrix, it obtains the similarities between all pairs of nodes, which is stored as matrix $M$. The final goal turns to reorder the rows and columns of $M$ so that inside each nontrivial diagonal block of matrix $M$, there exists at least one datapoint larger than a threshold for each row and column. It means that for each node in the diagonal block, there exists at least one other node densely connected with it. To accelerate the running speed, a hierarchical optimization approach is proposed in a top-down manner. The large generated diagonal blocks are finally regarded as extracted dense graphs. 

 \cite{qin2015locally} aims to retrieve top $k$ locally densest subgraphs (LDS) which are particularly defined in the paper. LDS is defined based on density (nodes within the LDS are densely connected) and compactness (removing a node in LDS will lose significant number of edges within it). With solid proofs, LDS  should satisfy a set of structural properties in compactness, cohesive and disjoint property. The original optimization approach is based on greedy algorithm, which runs in polynomial time. To reduce this running time, three optimization strategies are applied including pruning invalid nodes, optimization in finding dense subgraphs and optimization in verifying whether the dense graphs are eligible LDS.
 
OSLOM model \cite{lancichinetti2011finding} finds statistically significant communities through local optimization of a fitness function (significance score) which considers single cluster analysis and full network analysis. In single cluster analysis, the significance of a community is defined as the probability of finding the community in a random null model. In full network analysis, an iterative approach is leveraged to optimize the whole graph-level significance score by grouping nodes into proper communities. The approach first adds external nodes to existing subgraphs based on the generative probability. After that non-significant nodes are removed from the updated subgraphs.  

\cite{huang2014querying} regards a node as a query in the graph, and retrieves its k-truss communities through a designed compact and elegant index structure, which makes the overall model scalable with linear cost to serve online community search tasks.  On top of k-truss subgraph, k-truss community is defined with an extra edge connectivity constraint to ensure its node connectivity, which is any two edges in a k-truss community should either belong to the same triangle, or are reachable from each other through a series of adjacent triangles. To accelerate the running speed, a simple k-truss index are developed using an existing truss composition algorithm. Further an enhanced TCP-index is proposed to avoid two drawbacks from previous index: unnecessary access of dis-qualified edges and repeated access of qualified edges. 

Having the same research goal, \cite{zheng2017finding} takes edge weights into account to support better k-truss community detection for query nodes. It firstly utilizes a Breadth first search (BFS) method by removing unimportant edges iteratively. However this BFS approach suffers high computational cost which can not deal with large scale graph. To tackle this problem, it designs a KEP-index (Key Edges Preserved Index) which stores all possible k-truss community candidates in a space-efficient index structure. In the index construction process, communities are decomposed (edges are removed) in a hierarchical manner and the related key edges to reconstruct the hierarchical community tree are stored in the tree-shaped index. 

\cite{yan2019constrained} takes advantages of random walks to detect local graph communities. Starting from a set of seed nodes with given community memberships, the method labels each seed nodes with same/different colors according to its community. And a color-based random walk is applied to propagate colors across nodes to detect all other nodes communities. To store the color of each node, $K$ transition matrices are maintained where $K$ refers to the number of distinct colors in the seed nodes. Throughout an iterative propagation process, the color of each node can be learned and updated by its last-iteration color and the propagated colors from its neighbor nodes. The final color of the nodes are their deterministic community labels. 

Derived from k-core, \cite{li2015influential} proposes an online search algorithm to find the top k-influential communities in linear time through a linear-space index structure. A k-influential community means each node weight within should be at least $k$. The paper introduces a basic algorithm, a depth first search (DFS) and a final index based algorithm. The index based algorithm is able to handle large networks and runs in linear time. It uses a tree/forest index structure to hold all k-influential communities and utilizes a DFS function to iteratively add/remove nodes for index organization until all possible communities are scanned. 

\subsection{Community Detection Enhancement}

Enhancing community detection performance is also an essential task as it provides extra in-depths supports to explore how to improve model performance. As there is no main trend in this research topic, different types of approaches are introduced here with dispersive solutions, including assigning edge weights, removing unimportant nodes, or involving external information, etc. 

\cite{khadivi2011network} assigns proper edges weights on unweighted graphs to circumvent resolution limit \& extreme degeneracy problems and in the end enhances the performance of modularity based methods. modularity has been proved to suffer a resolution limit problem that the community size is constrained. With solid proofs, applying a weighted modularity can increase the upper bound and decrease the lower bound of community size. It allows to detect both larger and smaller communities using weighted modularity methods. Extreme degeneracy problem means that it goes more and more difficult to find the optimal community partition when the community size is large. And a proper weighting schema is able to strength the intra-community edges and weaken the rest to better clarify the community boundary. The weighting schema is defined as:

\begin{equation}
	W_{ij} = 
	\begin{cases}
	\frac{b_{ij}^{-\alpha}\cdot C_{ij}^{\beta}}{\sum_{k,m;k\neq m}b_{km}^{-\alpha}\cdot C_{km}^{\beta}},       & A_{ij} = 1\\ 
	0,  & A_{ij} = 0\\ 
	\end{cases}
\end{equation}
where $\alpha$, $\beta$ are positive numbers, $b_{ij}$ is the edge betweenness score and $C_{ij}$ is the common neighbor ratio between node $v_{i}$ and $v_j$.  \cite{de2013enhancing} proposes a similar weighting strategy that edges are weighted according to their betweeness centrality score. The proposed WERW-Kpath approximately estimates edge betweenness  score as the traversed probability of $\rho$ random walks with at most $k$ steps started from random nodes. This approximation reduces the NP hard edge weighting problem to a situation with worst time complexity as $\mathcal{O}(k|E|)$.

\cite{lai2010enhanced} shows a network preprocessing step can helps to alleviate the resolution limit problem for modularity based algorithms. It assigns random walks on graphs. By tracking these random walk trajectories, the similarity score of each pair of nodes are calculated as the updated weights in the original graph. The iterative steps run for several rounds and the modularity based community detection is applied in the last phase of graph.

\cite{wen2011improving} raises a point that hub nodes which are connected with many other nodes can disturb the community structure. Therefore, it proposes a degree-targeted node removal approach to reduce such noise. In detail,  it exhaustively searches through each node and introduces two types of approaches. First, the nodes with highest degrees are regarded as noisy nodes to remove. Second, the nodes whose removal causes the largest increase in modularity are regarded as noisy nodes to remove. 

\cite{he2016model} first calculates the structural similarity of each pair of nodes based on their degrees. A strong constraint matrix and a weak constraint matrix are both derived from node pairwise similarities. Further on, these matrices are incorporated into a stochastic block model to enhance the community detection. Intuitively, it aims to group each node in the same community with its high-similarity nodes and separate with low-similarity nodes in different communities. The parameters to be learned are optimized through a nonnegative matrix factorization approach. 

\cite{zhang2013enhanced} designs a semi-supervised algorithm to incorporate prior knowledge to guide the detection process. The prior knowledge utilized in the study contains must-link and cannot-link. It simply applies this information in the adjacency matrix $A$: If node $v_i$ and $v_j$ have a must-link, the weight of $e_{ij} = \alpha$; If node $v_i$ and $v_j$ have a cannot-link, the weight of $e_{ij} = 0$. It further extends to another refined adjacency matrix but involving a third node $v_k$: If $v_i$ and $v_j$ have a must-link and $v_i$ and $v_k$ have a must-link, then $v_j$ and $v_k$ should also have a must-link (the friend of my friend is my friend); If $v_i$ and $v_j$ have a must-link and $v_i$ and $v_k$ have a cannot-link, then $v_j$ and $v_k$ should also have a cannot-link (the friend of my enemy is my enemy). Conventional community detection methods can be directly applied on the refined adjacency matrix to get the enhanced community partition.

\cite{zhou2019adversarial} is the latest work which introduces two adversarial enhancement methods for community detection. One is called AE-GA, which is a genetic based adversarial algorithm to determine optimal edge modification and rewire community connections. Each chromosome in the genetic algorithm is a set of edges to be added or deleted in the graph. The fitness function is a variant of modularity:

\begin{equation}
f = \frac{\mathcal{Q}}{e^{|M_S - M_{real}|}}
\end{equation}  

where $\mathcal{Q}$ is current partition's modularity score. $M_{real}$ is the number of the ground-truth communities in graph $G$. $M_S$ is the number of communities detected by an particular method $S$. Another proposed model is called AE-VS model, which enhance the graph structure using node similarities. In detail, it rewires a graph by adding/deleting edges with a consideration of multiple similarity metrics and aggregates scattered communities to a better core community.

EdMot model \cite{li2019edmot} is a Motif-aware community detection. A motif-based hypergraph is constructed from the original graph to encode higher order node connections. Each of the top $K$ largest connected components, with the most number of nodes, is individually partitioned into communities through modularity-based methods. All the potential edges (generated from each pair of nodes) within each community are strengthened to put back to the original graph for enhanced community detection. 

\cite{veldt2019learning} presents a way to automatically learn the resolution parameters in community detection to control the size of communities. It is formulated to optimize a parameter fitness function which generally measures how these parameters can solve a generalized community question through a set of solid proofs. \cite{bhatt2019knowledge} incorporates hierarchical concepts of node attributes  into community detection. It detects node communities and simultaneously maintains the coherence of community label/context  summarized from node attributes. 

\subsection{Semi-Supervised Community Detection}
Conventional community detection approaches should be unsupervised and only leverage graph topological structures. However, unsupervised approaches are always weak in terms of model performance due to no-label guidance in the learning process. Therefore, to improve the model performance, partial information/constraints (must-links and cannot-links) are obtained to guide the learning process to the right direction, which forms the semi-supervised community detection task. 

SNMF-SS model \cite{ma2010semi} combines symmetric nonnegative matrix factorization and a semi-supervised approach where domain knowledge is incorporated to guide the detected communities. It provides two type of domain knowledge including must-links ($W_{ML}$) and cannot-links ($W_{CL}$). Each datapoint $w_{ij} \in W_{ML}$ means the edge between node $v_i$ and $v_j$ is given, and $w_{ij} \in W_{CL}$ means there should not be an edge between the two nodes. In the end, the paper aims to learn a low dimensional representation of nodes by optimizing the following objective function:

\begin{equation}
\argmin_{\hat{A} \geq 0, H \geq 0} || \hat{A} - HH^{T} ||^2
\end{equation}

where $\hat{A} = A - \alpha W_{ML} + \beta W_{CL}$ and $A$ is the adjacency matrix. $H$ is the low dimensional node matrix to be learned. 
 
SSNMF model \cite{liu2017semi} involves must-links $M$ as prior information into a nonnegative matrix factorization framework to alleviate the negative impacts from node degree heterogeneity.  In this model, two nodes having a must-link should be restrictively grouped into the same community using the learned low dimensional node representation matrix from NMF approach. It turns to minimize the representation difference between the two nodes having must-links. And the final objective function to minimize turns to be the weighted sum of must-link restrictions and original NMF objective function:
\begin{equation}
  \argmin_{\hat{A} \geq 0, H \geq 0} || \hat{A} - HH^{T} ||^2 + \lambda \sum_{ij}||h_i - h_j|| M_{ij}
\end{equation}
where $H$ is the low dimensional node representation matrix and $h_{i} \in H$ refers to the node $v_{i}$'s representation. 

PCSEO-SS model \cite{li2014extremal} is able to detect false connections and conflicted connections. In detail, it utilizes a dissimilarity index metric which describes the probability of two nodes belong to the same community. Higher value refers to less same-community probability. Prior node pairwise constraints (must-links and cannot-links) are corrected by setting up a threshold on the dissimilarity index.  Then an approach is given to maximize the modularity of the graph by removing extra-community edges iterative and in the end constructs the final community partition. 

\cite{cheng2014active} also makes full use of node pairwise constraints including must-links and cannot-links to extract high-quality communities. First, The two nodes in each cannot-link are forced to be grouped into different communities, which forms the skeleton of the overall community structure. Second, the communities containing nodes involved in must-links are merged into the same community without violating the existing skeleton community structure. After the nodes in must-links and cannot-links are processed, a greedy algorithm is applied to classify each of the remaining nodes into the existing raw communities. The nodes are selected and assigned to communities based on its similarity between each community and itself. 

\cite{yang2015active} proposes an iterative approach containing three steps. First, it applies nonnegative matrix factorization on the adjacency matrix to obtain the community structure. Second, for some pairs of nodes, their edges are uncertain but informative. The model designs a Connection Strategy by using human labeling on these edges to identify whether these edges should exist or not. Third, for these labeled new edges, a Disconnection Strategy is applied to highlight their importance weight, leading a result of  graph topological reconstruction. The three steps are optimized iteratively and the final round result will be regarded as the final community partition.

 \cite{jin2019graph} is the latest work by combining deep learning, Graph Convolutional Network (GCN), and statistical model, Markov Random Fields (MRF) to solve attribute graph community detection. The introduced end-to-end approach contains three layers. The first two layers are GCN layers which utilized both adjacency matrix $A$ and attribute matrix $X$ to learn the node community labels. The third layer is a MRF layer to refine the community partition result by enhancing node pairwise potential relationships. 
 
 WSCDSM model \cite{wang2018unified} contains two parts: A nonnegative matrix factorization approach with prior must-links to detect node communities and a semantic driven community detection via node content information. In the end, it integrates the two parts of the model into a unified framework and detects both topological and semantic communities. The training process combines the two individual objective functions together and uses a stochastic gradient descent method to optimize the two communities of each node.
 
 For other types of approaches, \cite{eaton2012spin} uses spin-glass model,  \cite{zhang2014phase} studies belief propagation and the stochastic block model, and \cite{yang2014unified} adds graph regularizations to penalize the dissimilarities between node pairs which should be in the same communities. 
 
\subsection{Summary}

I  formulate five different community detection tasks and introduce the best representative works of each task in the recent decade to demonstrate an overview. In fact, for each task, more and more research works having been published each year as the graph mining domain is with an increasing trend. Among the five tasks, overlapping community detection is the largest track which attracts the most researchers' attention for years. A rich number of papers are published regarding as this topic to the venues in computer science and physics domain. Semi-supervised approaches are also popular these years because empirical studies show that unsupervised learning approaches performs far worse than supervised ones. Moreover, beyond these five tasks, there are also many other tasks which focus on other perspectives towards community detection, such as Motif-based problems, math proofs, and edge community detection. In the future, more other interesting topics will be proposed, explored, and summarized. 
\section{Main Track Methods}

The main track of methods in community detection is quite clear according the published years of milestone papers. Modularity based papers are in dominant position in early 2000s after the concept of modularity is defined by \cite{newman2004fast} and \cite{newman2006modularity}. Spectral clustering methods start to raise in early 2010s, which aims to calculate eigenvectors from graph Laplacian matrix. \cite{nascimento2011spectral} is a representative survey paper summarizes a series of relevant researches. Matrix factorization based approaches are also very popular in the similar time period to spectral clustering, as both types of approaches utilize matrix decomposition techniques. Stochastic block model is a type of statistical inference model which has a lot of variants in the recent decade. Recently with the rapid increasing in deep learning, more models tend to detect communities in either an end-to-end fashion or by formulating conventional models under deep learning frameworks. In this section, the six discussed tracks are the most popular ones and related studies are summarized in following paragraphs.

\subsection{Modularity}

As the most important metric to measure the fitness of a community partition, modularity reflects the superiority of how much the communities preserves in-community edges better than a random partition model. Mark Newman published a series of papers regarding to explain the modularity from various perspectives, such as edge connections and adjacency matrix transformation. Among these papers, \cite{blondel2008fast} and \cite{newman2006modularity} are two representative works to explain what is modularity and how to find out a proper community partition which maximizes the graph modularity efficiently. 

Louvain method \cite{blondel2008fast} so far is the most efficient method to find the optimal partition with largest modularity. Within the paper, modularity is interpreted as the measurement to compare the density of within-community edges with that of between-community edges. It is calculated as:

\begin{equation}
	Q = \frac{1}{2|E|}\sum_{ij}(A_{ij} - \frac{k_ik_j}{2|E|})\delta(c_i,c_j)
\end{equation}

where   $k_i$ is the degree of node $i$.  $\delta(c_i,c_j) = 1$ if node $i$ and $j$ are in the same community, it equals to 0 otherwise.

The efficient Louvain method contains two phases and optimizes the modularity through an iterative approach. In the initialization step, all nodes are assigned to a single-node community. Then, for each node $i$, the paper considers to remove $i$ from its current community and plug into one of the communities which its neighbor nodes belong to. It will be re-assigned to a community which has the largest modularity gain. The second phase will run iteratively until no further modularity gain achieved. \cite{traag2019louvain} is a follow-up work of Louvain method to guarantee the generated communities are well-connected.

FPMQA model \cite{bu2013fast} is a parallel model to detect modularity based communities efficiently. The steps are similar as Louvain method. It also initializes a set of single-node communities, and merges communities which leads to the largest modularity gain in each step. The FPMQA model takes use of a mark array to store community states (busy or free) and merges communities based on their states in a parallel fashion.

In  \cite{newman2006modularity}, it theoretically proves that the original modularity optimization problem can be rewritten as the eigenvalue and eigenvector calculation on a defined modularity matrix. In a two-community detection scenario,  the modularity matrix can be written as:

\begin{equation}
Q = \frac{1}{4|E|}s^TBs
\end{equation}
where $s$ is a column vector in which $s_i = 1$ if node $i$ is in community 1 or $s_i = -1$ if node $i$ is in community 2. $B$ is a symmetric matrix in which $B_{ij} = A_{ij} - \frac{k_ik_j}{2|E|}$. In the end, after matrix transformations, The final goal is to find a proper community assignment $s$ to concentrate as much as positive eigenvectors of the matrix $B$.

A further approach, \cite{jiang2012modularity}, solves modularity maximization from the nonnegative matrix factorization (NMF) perspective. It runs on modularity Laplacian matrix instead of the modularity matrix.  

\cite{nicosia2009extending} extends the modularity to directed graphs for overlapping community detection. The modified modularity is calculated as:
\begin{equation}
	Q= \frac{1}{|E|}\sum_{c \in C}\sum_{i,j \in V}[r_{ijc}A_{ij}- s_{ijc}\frac{k_{in}k_{out}}{|E|}]
\end{equation}
where $r_{ijc}$ is the weight of contribution of edge between node $i$ and $j$ to the modularity of community $c$. And $s_{ijc}$ is the weight of contribution in the null model where communities are randomly assigned. The whole approach is optimized through a genetic process.

\cite{cafieri2011locally} introduces a local divisive heuristic to maximize graph modularity in undirected and unweighted graphs. It hierarchically divides the graph by using Kernighan-Lin heuristic, which proceeds a bipartition to reassign nodes from a community to the other. In each step, the bipartition reassignment which gains the largest modularity improvement will be selected. \cite{xiang2016local} also considers local graph connectivity and proposes a local modularity optimization to detect node communities.

 \cite{yang2016modularity} applies stacked auto-encode, a type of deep learning technique, to reconstruct the modularity matrix. Besides that, \cite{chen2014community} introduces several other variants of modularity, including modularity density,  fine-tuned modularity and fine-tuned modularity density.  Modularity density avoids the resolution limitation problem of original modularity. And the two fine-tuned variants improves the measurement by splitting and merging the graph structures. Derived from modularity density, \cite{sun2013maximizing} proposes a new measurement named modularity intensity to indicate the cohesiveness of community partition, which is defined as:
\begin{equation}
	M = \sum_{i=1}^{C}\frac{\alpha F(C_i,C_i) - \beta F(C_i,\bar{C_i})}{|C_i|}
\end{equation}
where $|C_{i}|$ refers to the number of nodes in community $C_i$. $F(C_s,C_t) = \sum_{\forall i\in C_s, \forall j\in C_t} A_{ij}\cdot B_{ij}$. And $\bar{C_i} = C - C_i$. $A$ is the adjacency matrix and $B$ is the edge weight matrix, which is calculated as:

\begin{equation}
	B_{ij} = \frac{\sum_{t=1}^{|V|}A_{it}\cdot A_{tj} + A_{ij}}{\sqrt{\sum_{t=1}^{|V|}A_{it} \cdot \sum_{t=1}^{|V|}A_{tj}}}
\end{equation}
 where $i,j \in V$ and $i \neq j$.
 

  
\cite{bagrow2012communities} shows a series of different tree and tree-like graphs, such as caley tree graph, z-ary graph and other clique or tree graphs. It theoretically proves that these types of graphs  always obtain communities which have high modularity scores. And adding nontree-like components in graph will destroy this phenomenon. 

To solve the resolution limit of modularity, \cite{zhang2013normalized} proposed a refined modularity metric by involving community degree factor, which compares the sum of average degree difference between the detected communities and randomly generated communities. The calculating formula is defined as:

\begin{equation}
Q = \sum_{k=1}^{K}\frac{\sum_{ij}(A_{ij} - \frac{k_ik_j}{2|E|})S_{ik}S_{jk}}{\sum_{i=1}^{|V|}S_{ik}}
\end{equation}

$S_{ik} = 1$ if node $i$ in community $k$, otherwise it equals to 0. Compared with the original modularitym the only difference is that  the normalized modularity divides the community size.

\cite{newman2016equivalence} is a very classic mathematical paper which proves the equivalence between modularity method and  degree corrected stochastic block model under special parameter settings.

On the contrary to modularity, \cite{chen2014anti} proposes an anti-modularity metric to detect anti-communities in graphs. Anti-community is a particular type of node partitionwhere nodes with nor or few connections within the community and densely connected with nodes from extra-communities. The anti-community is defined as:

\begin{equation}
Q = \frac{1}{|V|}\sum_{c \in C} \sum_{i,j \in c}(\sum_{k=1}^{|V|}a_{ik}a_{kj} -\frac{ k_ik_j }{|V|})
\end{equation}

where $\sum_{k=1}^{|V|}a_{ik}a_{kj}$ is the number of two-step paths between node $i$ and $j$ passing a third node $k$. The paper theoretically proves that the anti-modularity is fundamentally a principle component analysis method on the adjacent matrix. And in the end, the paper proposes a label propagation method to find a community partition with maximized anti-modularity. In detail, community labels are assigned to a set of seed nodes, which are propagated to other nodes through their connections. In the end, the nodes with same community labels are assigned to the same community.

\subsection{Spectral Clustering}
Spectral clustering is a type of approaches which leverages  graph matrices (modularity matrix, Adjacency matrix or Laplacian matrix) to find out the top eigenvectors or other graph characteristics so as to learn low dimensional representations for nodes. By reducing the dimension of sparse graph matrices, it can preserve denser graph information and explore more cohesive node relationships. Spectral clustering has a huge relationship with stochastic block models. \cite{lei2015consistency} proves that spherical k-median spectral clustering method can be extended the degree corrected stochastic block models. \cite{nascimento2011spectral} is a survey paper which clarifies some graph terminologies and introduces several graph cut and spectral clustering methods. However, as it is published over a decade ago, the mentioned models are a bit dated and lack of detailed model explanation.

\cite{newman2013spectral} is a theory paper shows that with a proper selection of parameters (i.e. how to choose the value of community membership matrix) in normalized cut via spectral method, it equals to the modularity based method and stochastic block model. \cite{bruna2013spectral} extends convolutional neural network (CNN) to general graph Laplacian matrix by adding a CNN operator on the eigenvectors of graph Laplacian. \cite{krzakala2013spectral} shows that spectral algorithms on a defined nonbacktracking matrix performs better than on adjacency matrix or other first-order approximate matrices. The nonbacktracking matrix $B$ is defined as:
\begin{equation}
B_{(u\rightarrow v),(w\rightarrow x)} = 
\begin{cases}
1,&   v = w,u \neq x\\ 
0,  & otherwise\\  
\end{cases}
\end{equation}

 \cite{saade2014spectral} introduces a spectral clustering method on Bethe Hessian matrix, which is also called as deformed Laplacian:
\begin{equation}
	H(r) :=(r^2-1)\mathbb{I}- rA + D 
\end{equation}
where $|r|>1$ is a regularized term. Particularly, $|r| = \sqrt{c}$ is used in this paper where $c$ is the average node degree in the graph. The eigenvectors of the matrix are calculated and those with negative eigenvalues in $H(\sqrt{c})$ or $H(-\sqrt{c})$ are selected. A standard k-means method is thereafter applied on these eigenvectors to generate node communities. Particularly, the negative eigenvalues of $H(\sqrt{c})$ shows the assortative communities, while those of $H(-\sqrt{c})$ represents the disassortative communities.

\cite{liu2013large} proposes a spectral method for large-scale graphs by generating supernodes connected to the regular nodes. To reduce the graph size, it generates supernodes and connects them to regular nodes. supernodes are regarded as cluster indicators to detect regular node communities. In the end, the original graph turns to be a bipartite graph containing two types of nodes. To construct $k$ supernodes, the paper firstly selects $k$ seed nodes, and calculates all the rest nodes' shortest distance to these seed nodes in order to group them into $k$ subsets as supernodes. The original graph can be converted to a bipartite graph to record the belonging of two node types where each row refers to a supernode and each column refer to a regular node. After that, a SVD approach is applied on the bipartite graph adjacency matrix to learn low dimensional representations for all nodes. In the end, a k-means clustering methods finally helps to detect node communities. 

\cite{nadakuditi2012graph} proposes a hierarchical and k-way spectral clustering method which decomposes top eigenvectors from constructed similar matrix $W$.  $W$ can be calculated as either a noisy hierarchical block matrix or a noisy k-Block Diagonal matrix. It proves the spectral clustering can tolerate the noises from similar matrix $W$ and still achieves good community partition even $W$ involves extra noise.   \cite{rohe2011spectral} summarizes the mathematical theory and proofs behind spectral clustering and stochastic block models. It also points out the the significance of spectral clustering for graph visualization. \cite{chaudhuri2012spectral} studies a spectral clustering method on graphs generated by Extended Planted Partition (EPP) model. EPP model generates graphs from a random graph with hidden community partition which affects edge generation probabilities. To facilitate the spectral clustering in such graphs, all graph nodes are randomly split into two sets $P$ and $Q$. Nodes in $Q$ are projected to the subspace generated by the bottom $k$ eigenvectors of random walk based graph Laplacian of $P$, and further community detection can group nodes in $Q$ to communities. The nodes in $P$ are partitioned in a similar way.

\cite{mahoney2012local} is a locally-biased spectral clustering method which involves extra constraints for community detection. It is formulated as: 
\begin{equation}
\begin{split}
&min \,\, x^T(D-A)x\\
&s.t.\quad  \left\{\begin{array}{lc}
x^TDx = 1\\
(x^TD\mathbb{I})^2 = 0\\
(x^TDs)^2 > k\\ \end{array}\right.
\end{split}
\end{equation}
where $s$ is the constraint matrix, $x$ is the node low dimensional representation aims to optimize. $k$ is a threshold parameter. It offers a feasible solution which satisfies the local constraints. 

\cite{zhang2015multiway} is a multiway spectral method maps modularity maximization to vector partitioning learning. Through a set of transformations, the paper finally aims to learn a representation $r_i$ for each node $i$ which maximizes the modularity score $Q$ in the graph where $Q$ is defined as:
\begin{equation}
	Q = \frac{1}{2|E|}\sum_{s=1}^{k}\big |\sum_{i \in s}r_i \big |^2
\end{equation}
 $s$ refers to each community and $i \in s$ means node $i$ in community $s$.

\cite{joseph2016impact} discusses the importance to choose regularization factors for community detection. It theoretically proves that the cluster discovery result is not fully depend on the minimum node degree. And regularization can better support to group low-degree nodes to well-structured communities. In this paper, Eigengap is defined as the gap of the $k$ smallest eigenvalues to the rest of eigenvalues, which can be controlled by the regularization factors and in return will affect spectral clustering performance.

SCORE model \cite{jin2015fast} uses the coordinate-wise ratios between the largest eigenvector and the rest largest eigenvectors to construct a decomposed matrix for clustering, which significantly reduces the nuisance from degree heterogeneity problem. TSC model \cite{benson2015tensor} is a spectral method on high-order graphs where graph information beyond edges implicitly connects nodes. Instead of finding out the pairwise relationship between nodes, it aims to detect node triplet relationships. Therefore, instead of constructing a 2-D matrix for matrix decomposition, it builds up a 3-D matrix to hold node transition probabilities. 

SClump model \cite{li2019spectral} proposes a spectral clustering for heterogeneous graphs. It constructs a similarity matrix based on metapaths from \cite{sun2013pathselclus}. \cite{sun2013pathselclus} calculates pairwise node similarities for each metapath $P_i$, and sum them over to construct the metapath similarity matrix $\sum_{i} \lambda_i P_i$. Thereafter, it aims to learn a refined similarity matrix $S$ that exhibits a clear clustering structure, the final objective function turns to be:
\begin{equation}
\min ||S-\sum_{i} \lambda_i P_i||^2_F + \alpha ||S||^2_F + \beta ||\lambda|| + 2 \gamma\sum_{i}^k\sigma_i(L_S)
\end{equation}
which is constraint with $\sum_{j=1}^n S_{ij} = 1, S_{ij} \geq 0$ and $\sum_{i}\lambda_i = 1, \lambda_i \geq 0$. $\sigma_i(L_S)$ is the $i_{th}$ smallest eigenvalue of Laplacian graph $L_S$ for matrix $S$.

\cite{mercado2019spectral} extends spectral clustering for signed graphs by constructing a family of Signed Power Mean Laplacians, which is defined by a transformation function of the normalized Laplacian graphs generated from both positive edges and negative edges. \cite{wu2018scalable} is a scalable method utilizes graph random binning features (RB) and CoreCut model \cite{zhang2018understanding} introduces the relationship between regularized spectral clustering and graph conductance minimizing. 


\subsection{Stochastic Block Model}

SBM is a major track of community detection method and there are so many papers regarding to this topic. As a type of approaches derived from statistical inference, most of papers in this track are theory papers aiming to prove their model efficiency under certain scenarios such lower \& upper bound limitation. Hereby, I will broadly introduce the general goal in each paper instead detailed theoretically proofs. \cite{karrer2011stochastic}, as one of the most classic papers which introduces the SBM concept as a generative random graph model to reconstruct the original graph from communities. It assumes there exists community partition $C$ and expected edge weight $w_{rs}$ between node $i$ in community $r$ and node $j$ in community $s$ following Possion distribution. After a set of transformation functions, the final generative probability to maximize turns to be:
\begin{equation}
\log P(G|w,C) = \sum_{rs} (m_{rs}\log w_{rs}-n_rn_sw_{rs})
\end{equation}
where $m_{rs} = \sum_{ij}A_{ij}\delta{g_i;r}\delta{g_j;s}$ refers to the total number of edges between groups $r$ and group $s$ and $n_{r}$ is the number of edges in group $r$. 
The paper also proves the equivalence between the standard SBM and modularity for undirected graphs. It further extends to a degree-corrected SBM  to solve the limitation caused by node degree heterogeneity. 

\cite{abbe2017community} is the latest survey paper published in 2017. It is a very detailed paper which introduces the general forms of stochastic block model. The main content discusses the thresholds during SBM phase transitions for exact recovery, weak recovery and partial recovery. It also extends to other related tracks of approaches such as graph-splitting, semi-definite programming and spectral clustering. In the end, it points out several open questions such as possible extensions for semi-supervised or dynamic graphs. 

\cite{mossel2014belief} is a theory paper which discusses the graph reconstruction problem of sparse SBM with only two communities. The inter- and intra- edge connection probability are $a/n$ and $b/n$ where $a,b$ are parameters and $n = |V|$ is the number of nodes. The paper proposes a belief propagation model to correctly assign node community labels under a general situation where $(a-b)^2> C(a+b)$ where $C$ is a constant. \cite{mossel2016density} also discusses the same binary SBM where there are only two communities in the graph. It converts the original community recovery problem to a transformed tree reconstruction problem. Particularly, it analyzes the density evolution of belief propagation on trees with Gaussian approximations.

\cite{peixoto2012entropy} is an ensemble SBM model containing several existing SBM variants. It uses an entropy based log-likelihood function to infer community structure under two different scenarios including a soft constraint and a hard constraint. For a soft constraint, each imposed node degree refers to its average value among all SBM variants. While for a hard constraint, each imposed node degree should be exactly same among all variants.

\cite{latouche2011overlapping} is an overlapping SBM  which associates a latent vector for each node following multivariate Bernoulli distribution. The edge generative probability is calculated based on the latent factor of its start node and end node. \cite{abbe2015community} talks about the current limitation in partial or exact recovery of SBM and generalizes the discussion to overlapping communities.

\cite{qin2013regularized} involves degree-corrected SBM to generate adjacency matrix which are used for spectral clustering. With proper adjustments in parameter selection, the regularized spectral clustering can achieve better performance for graphs where node degree significantly varies. It also points out choosing a parameter close to average degree can balance performance difference from several competing models. \cite{zhang2016minimax} is a theory paper which proposes a minimax rate to discuss the lower and upper bound for exact or partial recovery in SBM.

\cite{zhao2012consistency} checks the performance consistency for degree-corrected SBM. By comparing a set of different models, it proves that degree-corrected SBM can always obtain a stable model performance. Modularity based methods need to have special parameter constraints in order to achieve a consistent result while likelihood-based methods do not. \cite{celisse2012consistency} is another paper checks the consistency of maximum-likelihood and variational estimators in SBM. It proves that in SBM variational estimators can be asymptotically equivalent to maximum-likelihood estimators to estimate the edge appearance probability between nodes. \cite{yan2014model} focuses on the model selection in SBM as there are too many parameters to tune and general model-selection criteria can't properly work. The paper discusses the influence of different log-likelihood ratio distributions in both standard SBM and degree-corrected SBM. It further extends to sparse graphs to explore more graph scenarios. Moreover, it proposes a belief propagation based linear-time approximations for log-likelihoods, which proves to have relatively satisfied agreements. 

\cite{yun2016optimal} introduces a new labeled SBM task where there are labels appeared with a certain probability $p(l,i,j)$ between two nodes in community $i$ and $j$. In the end, the paper proposes a spectral partition method under SBM to reconstruct the communities from the observation of these appeared labels. 

\cite{peixoto2014efficient} utilizes an optimized Markov chain Monte Carlo (MCMC) method to efficiently infer SBM in large graphs. It heuristically defines a node moving probability for community $r$ to $s$ as:
\begin{equation}
p(r \rightarrow s| t) = \frac{e_{ts}+ \epsilon}{e_{t} + \epsilon B}
\end{equation}

where $t$ is the community label of a random node, $e_{t}$ is its degree. $\epsilon$ is a tuning parameter and $B$ is a relatively large constant. 

Given a random labeled graph generated by standard SBM, \cite{xu2014edge} aims to infer its edge community distributions from node latent attributes. specifically, it proves that no model works well without observation if average node degree is below a certain threshold.  

\cite{he2015stochastic} proposes a SBM based edge community detection task and addresses the community size heterogeneity problem which is often ignored by previous works. In its definition,  an edge $<i,j>$ is generated by two nodes  $i,j$ selected from community $z$. The node selection probabilities are $\theta_{iz}$ and $\theta_{jz}$ and  community size is $w_z$. In its generative model, a community $z$ is firstly chosen with $w_z$ nodes. And node $i$ and $j$ are selected in this community with the aforementioned probability to form an edge. The final expected number of edges between the two nodes in community $z$ are calculated as:
\begin{equation}
\hat{A}^z_{ij} = w_z\theta_{iz}\theta_{jz}
\end{equation}
And the final expectation of edge number between node $i$ and $j$ is the sum over all possible communities as $\hat{A}_{ij} = \sum_{z}\hat{A}^z_{ij}$.


\cite{wang2017likelihood} is a likelihood based SBM  which can be extended to degree-corrected SBM with proper parameters. It calculates the asymptotic distribution under overfitting and uderfitting situation and proves its result performs stable when average node degree grows  at a polylog rate.  \cite{lei2016goodness} is a goodness-of-fit test on SBM to offer a baseline result for comparisons with other competing models. \cite{gao2018community} is an in-depth exploration particularly for degree-corrected SBM and proposes a polynomial time algorithm for asymptotic optimization in  the degree-corrected SBM. 


%,\cite{heimlicher2012community},\cite{yun2014accurate},
%
%\cite{peixoto2017nonparametric},\cite{sarkar2015role},\cite{lyzinski2014perfect}


\subsection{Deep Learning}

In recent years, deep learning techniques are more and more involved in community detection tasks. Related approaches either leverage graph neural networks (GNN), graph convolutional networks (GCN) or other standard deep frameworks (i.e. autoencoder) to learn community embeddings or direct node-community distributions. 

Autoencoder is a very intuitive approach to compress node original one-hot encoding to a latent low-dimensional space. By natural, the node latent vector can be regarded as its community distribution or be further clustered by other community detection models. \cite{huang2014deep} is the first work which uses autoencoder technique to learn node embeddings. It uses a four fully connected layers as the encoder to project original input to a latent space. Then a symmetric decoder is processed on the latent vector to reconstruct the original input. A locality-preserving constraint (from k-nearest neighbor nodes) and a group sparsity constraint (from same-community nodes generated from group lasso) are combined with the reconstruction error as the final objective to minimize. In the end, k-means clustering method is applied to the latent node representation for community detection. Similarly \cite{tian2014learning} is a very classic method to learn node non-linear embedding via stacked autoencoder, and applies k-means to obtain node communities afterwards. \cite{sun2017non} is another similar approach but with a shared weight in both encoder and decoder. And the final optimized weight is regarded as the node-community matrix. MGAE model \cite{wang2017mgae} is a marginalized autoencoder which leverages both graph structure and content information into a unified GCN framework. The model considers both graph adjacency matrix and node content to learn latent node representations. And a spectral clustering is taken on top of the representations to detect the final node communities.

\cite{cavallari2017learning} jointly models three tasks to learn node embeddings, community embeddings and detect communities. The community embedding are generated from multivariate Gaussian distribution. And a Gaussian Mixture Model (GMM) helps to learn the generative probability of a node from a community. Meanwhile, the node embedding is optimized from a skip-gram model with negative sampling with the goal to generate nodes from its neighbor nodes.  Similarly, \cite{sun2019vgraph} is also a multi-task generative model to jointly learn node embedding and detect communities. In its assumption, a node is represented as a mixture of communities, and a community is a multinomial distribution over all nodes.  The generative probability of a neighbor node $u$ of node $v$ is calculated as:
\begin{equation}
p(u|v) = \sum_{c}p(u|c)p(c|v)
\end{equation}
The probability is the sum over all conditions that node $v$ generates a community $c$ first, and community $c$ thereafter generates node $u$. The whole generative process can be optimized under a deep framework using stochastic gradient decent optimization.

DLC model \cite{shao2015deep} proposes a single layer transformation as:
\begin{equation}
\min_{W,C} ||A-WDC||^2_F + \lambda||C||^2_F
\end{equation}
where $W$ is the linear transformation function, $D$ is the dictionary in the linear coding, $C$ is the code of the graph nodes. This linear transformation can be stacked into multi-layers to achieve a deep linear coding schema. The paper also theoretically proves its equivalence to the marginalized denoising autoencoding with spectral clustering.  

ComE model \cite{zheng2016node} proposes an interesting task to learn community embeddings instead of node embeddings. Instead of representing communities as a vector and inspired by the Gaussian mixture model, ComE formulates that community embeddings follows multivariate Gaussian distribution with a tuple of a mean vector and covariance matrix. Therefore, the main output of this model is a set of community embedding $(\psi_k,\Sigma_k)$ for each community $k \in \{1,..,K\}$. Node embeddings $\phi$ are learned as a prior knowledge via LINE method \cite{tang2015line} to support community embedding. The whole process is optimized as a generative model which uses community embedding to generate node embeddings $p(v_i|c_i=k,\phi_i,\psi_k,\Sigma_k)$.

GEMSEC model \cite{rozemberczki2019gemsec} jointly learns node embeddings and clusters nodes into communities. It uses a negative sampling to maximize the generative probability of neighbor nodes given the current node. Meanwhile, it adds a community cost to the node embedding objective to learn community centers:
\begin{equation}
\argmin \sum_{v\in V}\big[ ln \Big(\sum_{u\in V}exp(f(v)\cdot f(u)) \Big) \big] + \gamma \sum_{v \in V}\min_{c\in C}||f(v)-u_c||_2
\end{equation}
where $f(v)$ is the embedding of node $v$. $N_S(v)$ is a collection of nodes within a window size towards node $v$ through random walks. The first term is the node embedding learning cost, and the second term is the community center distance cost. 

DFuzzy model \cite{bhatia2018dfuzzy} is a three-step approach to learn fuzzy clusters. In the beginning, a pre-training step learns the initialized community centers through a personalized PageRank. And an autoencoder approach is taken with the PageRank result to learn the initialized node communities regarding as the center nodes. Second, modularity is used to redefine the partition result of initialized communities. Third, the community centers are updated based on the distance of the rest nodes in the community, which are calculated from the last layer of the model. The whole process will iteratively updated until convergence.

\cite{yang2016modularity} learns node embedding from deep neural network, and extends its model to a semi-supervised community detection task by involving pairwise node contraints. The node embedding learning process is a standard stacked autoencoder approach to firstly project each node to a latent space and reconstruct it after that. To involve the pairwise node contraints, the objective function adds $\lambda Tr(H^TLH)$ to the reconstruction error in the stacked autoencoder. $H$ is the learned node low dimensional vector, which is also regarded as node-community distribution. $L$ is the Laplacian matrix of the node constraint matrix. 

\cite{bruna2017community} is a very first work to really apply graph neural network (GNN) techniques in community detection tasks. It is a stacked model which contains multiple stacked components. In each layer of the component, it involves adjacency matrix in the non-linear transformation and the final output is node community labels. The whole approach can be optimized in an end-to-end fashion. 

Cluster-GCN model \cite{chiang2019cluster} is an efficient model to leverage graph convolutional network (GCN) for community detection. It makes an innovation for the mini-batch SGD update by sample a subgraph to optimize the corresponding node embeddings. This strategy significantly reduces the computational cost in the orignal GCN framework without losing any graph information. 

MRFasGCN  model \cite{jin2019graph} solves a very complex task: using both graph convolutional network (GCN) and markov random field (MRF) for semi-supervised community detection in attribute graphs with semantic information. The model input contains adjacency matrix $A$, node attribute matrix $X$ and node similarity matrix $K$ calculated from $A$ and $X$. The first two layers are GCN layers only involved with $A$ and $X$ and reLU activation function. In detail, the first layer is represented as $AXW^{(0)}$, the second layer is represented as $AX^{(0)}W^{(1)}$.  And the third layer takes $K$ into account as $KX^{(2)}W^{(2)}$. The last layer result is passed a MRF layer to learn the pairwise constraint between nodes.  

For other works, \cite{zhang2019attributed} exploits the high-order graph convolutional networks and theoretically discusses the order influence to the model performance in attribute graph clustering. DMGC \cite{luo2020deep} is the latest work which detects communities in multi-graphs simultaneously via an attention module and minimum-entropy loss.  CommunityGAN \cite{jia2019communitygan} jointly learns node embeddings and detects overlapping communities through a generative Adversarial net (GAN). The generator aims to generate motifs from nodes to approximate the real graph, while the discriminator aims to detect which is the fake motif generated from the generator or the ground-truth motif.

\subsection{Matrix Factorization}
A lot of matrices can be derived from graphs to reveal its structure, some of the most popular ones are degree matrix $D$, adjacency matrix $A$, Laplacian matrix$L = D-A$, normalized Laplacian matrix $D^{-1/2}LD^{-1/2}$, stochastic random walk matrix $Q=AD^{-1}$ and modularity matrix $M_{uv} = A_{uv}-d_ud_v/2|E|$. Therefore, matrix factorization by nature can be directly utilized on these matrices to learn a hidden representation on nodes. Intuitively, if each dimension of the representation is regarded as a community, the node vectors can also indicate their affiliations in each community to solve the overlapping community problem. Inspired by this, nonnegative matrix factorization has been a popular track of methods for years.

\cite{wang2011community} is the first work that utilizes nonnegative matrix factorization (NMF) for community detection. In its paper, it proposes three NMF based techniques including Symmetric NMF, Asymmetric NMF and Joint NMF. The Symmetric NMF is the simplest form of graph NMF, which assumes the graph is undirected so that adjacency matrix $A$ is symmetric. The objective of Symmetric NMF turns to learn the low dimensional representation of nodes, which refers to the probability of all nodes belonging to communities:
\begin{equation}
		\min_{X \geq 0} ||A-XX^T||^2_F
\end{equation}
In a directed graph, Asymmetric NMF is used to learn a node community membership matrix $X$ and a diagonal matrix $S$ which shows the connectivity within each community. Therefore, the objective function has a small change compared with Symmetric NMF method:

\begin{equation}
\min_{X,S \geq 0} ||A-XSX^T||^2_F
\end{equation}

The Joint NMF considers an even more complex scenario which considers a heterogeneous graph whose adjacency matrix $A$ refers to the connections between the two types of nodes. $U$ and $D$ refer to the connections within each individual node type. Therefore, to learn a latent matrix $X$ to uncover the belonging relationships of two type of nodes, it considers the all three aforementioned information:  

 \begin{equation}
 \min_{X,\alpha, \beta \geq 0} ||A-X||^2_F + \alpha ||U-XX^T||^2_F + \beta ||D-X^T X||^2_F
 \end{equation}
 
SymNMF model \cite{kuang2012symmetric} comprehensively explains how to use the NMF to graph clustering. Derived from the standard NMF which decomposes on the adjacency matrix, it introduces another similar matrix $D^{-1/2}AD^{-1/2}$ and explains its equivalence to the Normalized Cut method. The paper also generalizes the similar matrix to a kernel function as $\phi(X)\phi(X)^T$. \cite{kuang2015symnmf} is a follow-up work on the same SymNMF model but with more detailed supplementary. BIGCLAM \cite{yang2013overlapping}, a previously mentioned overlapping community detection model, is also leverages nonnegative matrix factorization techniques.

\cite{yang2012clustering} uses multi-step random walks to calculate node pairwise similarities. The original similarity matrix is the normalized Laplacian matrix $Q=D^{-1/2}AD^{-1/2}$ where $D$ is the diagonal degree matrix. The $j$ step node random walk similarity matrix is calculated as $(\alpha Q)^j$ where $\alpha$ is a decay factor. Summing over all possible $j$ steps. the limitation is calculated as $\sum_j^{\infty}(\alpha Q)^j = (I-\alpha Q)^{-1}$. Therefore, the all step similarity matrix is used to replace the adjacency matrix in the objective function. The goal is to learn a low dimensional node-community distribution $W$ by minimizing the reconstruction error to the similarity matrix:
\begin{equation}
\min_{W \geq 0} ||c^{-1}(I-\alpha Q)^{-1} - WW^T||^2_{F}
\end{equation}
where $c=\sum_{ij}[(I-\alpha Q)^{-1}]_{ij}$ is a normalizing factor.

\cite{tang2014uncovering} proposes a two-step matrix factorization approach. First, a singular value decomposition on the adjacency matrix is calculated as $A= U\Sigma V^T$. It selects the top ranked columns in the three matrices to get a refined adjacency matrix $N$ with more condensed information. A Bayesian Nonnegative Matrix Factorization (BNMF)  \cite{psorakis2011overlapping}  is thereafter applied on the matrix $N$ by assuming each node in $N$ follows a Poisson distribution. In detail, it calculates the posterior distribution of two smaller matrices $W$ and $H$ by maximizing the posterior criterion:
\begin{equation}
\min_{W,H,\beta \geq 0} p(W,H,\beta|N)
\end{equation}
where $\beta$ is a scale hyperparameter with Gamma distribution, and $W,H$ are both with half-normal probability distributions.

BNMTF \cite{zhang2012overlapping} is a bounded nonnegative matrix factorization approach which uses tri-factorization to decompose adjacency matrix $A$ into three sub-matrices $U,B,U^T$.  $U$ is the node-community matrix where each row refers to a node distribution among all communities, which is bounded between 0 to 1. $B$ is the connections between communities. The goal is to use the three matrices to regenerate a matrix $\hat{A} = UBU^T$ to let $A \approx \hat{A}$. The reconstruction error can be measured using either square loss $l_{sq}$ or KL-divergence $l_{kl}$, which are calculated as :
\begin{equation}
\begin{aligned} 
l_{sq}(A,U,B) = ||A-UBU^T||^2_F \\
l_{kl}(A,U,B) = \sum_{ij}(a_{ij}ln\frac{a_{ij}}{\hat{a_{ij}}} - a_{ij} + \hat{a_{ij}})
\end{aligned} 
\end{equation}

where $a_{ij}$ is the related datapoint in $A$ and $\hat{a_{ij}}$ is the estimated value of $a_{ij}$.

SBMF \cite{zhang2013overlapping} not only detects overlapping communities via a binary matrix factorization, but also detects node outliers from unweighted graphs. Given its binary adjacency matrix $A$, $A_{ij} = 0$ if there is no edges between node $i$ and $j$, otherwise $A_{ij} = 1$. The paper aims to find a binary community matrix $U$ where $U_it = 1$ if node $i$ in community $t$ and 0 if not. If a node $i$ belong to multiple communities, the sum of its vector will be larger than 1 ($\sum_{t} U_{it} > 1$). And $\sum_{t} U_{it} = 0$ if the node $i$ is an outlier. In its model, it assumes there are a few outlier nodes which do not belong to any community, and the outlier nodes should be as few as possible. Therefore, from the basic matrix factorization objective function, it adds a term regarding the outlier node penalty and uses L1 normalization in the matrix factorization objective:
\begin{equation}
\min_{U} ||A- UU^T||_1 + \sum_i [1-\Theta(\sum_j U_{ij})]
\end{equation}
 where $\Theta(X)$ equals to 1 if $X>0$ or 0 if $X \leq 0$.
 
\cite{liu2017semi} is a semi-supervised NMF approach which involves prior information (must-links $l_{ml}$) into the NMF objective function. The prior information constructs a constraint matrix $M$ where 
\begin{equation}
M_{ij} = 
\begin{cases}
1,&   i = j\\ 
0,  & (v_i,v_j )\in l_{ml} \\ 
\epsilon,  & others\\ 
\end{cases}
\end{equation}
The constraint matrix will be involved in the conventional objective function to learn the node-community matrix $X$:
\begin{equation}
\min_{X} ||A-XX^T||^2_F + \frac{\lambda}{2}\sum_{ij}||x_i - x_j||^2M_{ij}
\end{equation}
where $x_i$ is the $i_{th}$ row in node-community matrix $X$.  \cite{shi2015community} is a similar semi-supervised approach on unweighted, undirected graphs but with both must-links and cannot-links constraints. 

Graph regularization is a typical strategy used on top of constructed graph matrices for enhancing the performance of matrix factorization models.  DNMTF \cite{shang2012graph} is a novel graph dual regularization non-negative matrix tri-factorization model which considers the graph regularized terms from both structure based and feature based perspectives. A k-nearest neighborhood method helps to construct a data graph from graph topological structure,  and a feature graph from node feature information. The objective function therefore includes three components:
\begin{equation}
	\min_{U,S,V \geq 0} = ||A- USV^T||^2_F + \lambda Tr(V^TL_V V) + \mu Tr(U^T L_UU)
\end{equation}
where $U,S,V$ are matrices need to be learned, $L_v$ and $L_U$ are related graph Laplacian matrices for data graph and feature graph. $Tr(\cdot)$ is the trace of related matrix. $\lambda$ and $\mu$ are weights for the two regularized terms.

NMTF \cite{pei2015nonnegative} utilizes three types of graph regularizations to capture user similarity, message similarity and user connections seamlessly in social networks. It contains three binary graph matrices ($M_{u-u},M_{u-f}, M_{t-f}$ ), two similarity matrices ($S_{u-u},S_{t-t}$) and one binary interaction matrix ($R$). The goal of this paper aims to learn a user binary cluster matrix $U$, message binary cluster matrix $V$ and word binary cluster matrix $W$. By involving conventional NMF approach with three regularized terms, the overall objective function is defined as:
\begin{equation}
\begin{aligned}
\min_{U,V,WH_1,H_2,H_3}||M_{u-u}-UH_1U^T||^2_F + ||M_{t-f}-VH_2W^T||^2_F + ||M_{u-f}-UH_3W^T||^2_F + \\
\alpha Tr(U^TL^uU)+\beta Tr(V^TL^tUV) + \gamma  Tr(U^TL^\tau U)
\end{aligned}
\end{equation}
$L^u,L^t,L^\tau$ are the Laplacian matrix of $S_{u-u},S_{t-t},R$. And $H_1,H_2,H_3$ are three matrices to be optimized. To ensure a user/message can only belong to one cluster, the objective function should be constraint by $UU^T=I$ and $VV^T=I$.

MHGNMF model \cite{wu2018nonnegative} extends the graph regularizations to hypergraphs by considering high-order node information to enhance model performance. RGNMF \cite{huang2018robust} considers an extra error matrix $S$ in the conventional NMF approach:
\begin{equation}
\min_{U,V,S} ||A-UV^T -S||^2_F + \gamma ||S||_1 + \mu \sum_{ii^{'}}  W^U_{ii^{'}}||U_i-U_{i^{'}}||_2 + \lambda W_{jj^{'}}^V||V_j-V_{j^{'}}||_2
\end{equation} 
The last two terms are related graph regularized terms. 

DANMF model \cite{ye2018deep} is a deep autoencode-like method to learn node communities via an encoder-decoder component. It is a very straightforward method which uses multiple layers to transform the original adjacency matrix to a low dimensional community space (decoder component). Then it uses the node-community matrix to reconstruct the original adjacency matrix (encoder component). The overall loss it the combination of two component errors with a regularization term. M-NMF model \cite{wang2017community} incorporates a NMF based node representation learning model and a modularity based community model together to optimize them jointly.

\subsection{Flow-Based}
Flow-based models assumes either random walks or information propagation in the graph. Through these flow-like process, the energy or information of each node will be propagated to its neighbor nodes. When this process becomes stable, the nodes contains similar type or amount information will be grouped into the same community. By nature, flow-based models are all Markov chains, as the next step node will be fully dependent on the current step node.  This type of approaches are quite scattered, random walk models, local search models, Potts model, and heat kernel based models all fall into this category. 

Map Equation \cite{rosvall2008maps} is a metric to quantify how well a community can compress graph information. Derived from this metric, there are several variants are proposed to serve other more complex scenarios such as hierarchical community detection \cite{rosvall2011multilevel}, dynamic community detection \cite{rosvall2014memory} and sparse Markov chain to solve vast parameters problem in  high-order Markov chain \cite{persson2016maps}. Particularly, \cite{rosvall2014memory} shows second-order Markov dynamics in the random walks can lead to significant influence for community detection, node ranking and information propagation.  

In detail, \cite{rosvall2008maps} uses a probability of random walks as an proxy of information flow in the graphs by introducing the Map Equation metric. In order to describe a random walk path, it utilizes huffman codes to encode nodes by assigning shorter codewords with hub nodes and longer codewords to rare nodes. And a path can be described as the concatenation on the codewords of all nodes appeared in the path. The goal in this approach aims to find an optimized community partition $C$ which group all nodes into $k$ communities, which has the minimum average description length of all the paths $L(M)$. The metric is defined as:
\begin{equation} 
 L(M) = q_{\curvearrowright}H(\mathcal{L})+\sum_{i=1}^{k} \textit{$p_{\circlearrowright}^{i}$}H(\textit{$\mathcal{P}^{i}$}) 
\end{equation}
This metric contains two parts: first is the entropy $H(\mathcal{L})$ of the random walks between communities, and second is the entropy $H(\mathcal{P}^{i}) $ of random walks within each community (where exiting current community also is considered a step in random walk). To minimize the Map Equation metric, a deterministic greedy search algorithm is proposed and refined via a simulated annealing process. 

 \cite{rosvall2011multilevel} extends the original map equation to a hierarchical version. For a hierarchical map $M$ with $k$ communities, each community $i$ contains a submap $M^i$ and $m^i$ sub-communities, the hierarchical map equation is calculated in a nested way:
 
 \begin{equation}
 L(M)= q_{\curvearrowright}H(\mathcal{L})+\sum_{i=1}^{k} L(M^i)
 \end{equation}
 
UEOC model \cite{jin2011markov} uses Markov random walks with constraints strategy to detect overlapping communities. It contains four steps: first, select the node with maximum degree and non-community membership; second, apply random walks on the selected nodes with a constraint number of steps. Calculate the probability of each node being the end node and rank them; third, select nodes with conductance score larger than a threshold to be in the same community of target node; fourth, if there are still nodes without being assigned to at least one community, repeat the step 1 until no such nodes remained. 

\cite{kloster2014heat} introduces a heat kernel based diffusion model to detect communities in a local and deterministic manner. The overall goal of this approach is to approximate a heat kernel $h$ as a diffusion function in the graph:
\begin{equation}
h = e^{-t} \left (\sum_{k=0}^{\infty} \frac{t^k}{k!}(P)^{k}\right)s 
\end{equation}
where $P=AD^{-1}$ is the random walk transition matrix and $D$ is degree matrix. $t$ is a coefficient term. And $s$ is an initialized seed vector which sums to one. In the end, the subgraphs with low conductance score will be regarded as single communities.

\cite{zlatic2010topologically} studies a particular Topologically Biased Random Walk (TBRW) on graphs for community detection. A standard transition matrix $T$ is calculated as:
\begin{equation}
T_{Ij} = \frac{W_{ij}}{\sum_{l} W_{lj}}
\end{equation} 
$W_{ij}$ is the edge weight between two nodes $i$ and $j$. In a biased random walk, the edge weight can be defined as $W_{ij} = A_{ij}e^{\beta x_i}$. $x_i$ can be any defined factors such as related node degree. After the transition matrix is calculated, spectral clustering can be leveraged to select top eigenvectors as node community distributions.

\cite{liu2010detecting} proposes two simulated annealing algorithms,SADI (dissimilarity-index-based) and SADD (diffusion-distance-based), to generate communities with maiximized modularity under a k-means framework. A dissimilarity index measures the proximate extent between each pair of nodes. And diffusion distance metric calculates the distance difference for each pair of nodes to the rest nodes in the graph.  Each community centroid is calculated given a certain community partition and related metrics. After that, a simulated annealing k-means function is applied to find the best partition with largest modularity based energy score.  Nodes will be merged or separated in an recurrent way according to the gain or loss of the graph modularity.

\cite{li2012potts} uses Potts model via a Markov process which assumes nodes as spins and simulates spin dynamic value changes trough spin-spin correlations.  The detected dynamics are used to detect node hierarchical communities. \cite{lambiotte2012ranking} ranks nodes and detects their communities via PageRank algorithm with teleportation. Instead of standard teleportation to nodes (the next step of a random walk has a small chance to a random-selected node instead of direct-connected nodes), this paper proves the teleportation to edges can achieve a more robust community partition result.

\cite{wang2013fuzzy} is a fuzzy overlapping community detection method based on distance matrix calculated via local random walks. A $t$ step local random walk (LRW) index between two nodes $i$ and $j$ is defined as:
\begin{equation}
s_{ij}^{LRW}(t) = \frac{k_i}{2|E|}\cdot \pi_{ij}(t) + \frac{k_i}{2|E|}\cdot \pi_{ji}(t)
\end{equation}
where $k_i$ is the node degree of $i$, $ \pi_{ij}(t)$ is the probability of a $t$ step random walk with start node $i$ and end node $j$. Derived from local random walk index, the $t$ step superposed random walk (SRW) index between $i$ and $j$ is calculated as $s_{ij}^{SRW}(t) = \sum_{l=1}^{t}s_{ij}^{LRW}(t)$. Subsequently, each datapoint $d_{ij}$ of the distance matrix $D$ is calculated as:

\begin{equation}
d_{ij}=
\begin{cases}
1- s_{ij}^{SRW}(t),    & \quad  i \neq j\\ 
0,  & i = j\\ 
\end{cases}
\end{equation}

After that, a standard multidimensional scaling (MDS) method is applied on $D$ to project each node into a low dimensional space. An existing fuzzy-means (FCM) method helps to learn each node fuzzy communities from the projected space.

\cite{yang2014closed} discovers that three to four steps closed walks in the graph can reveal community structures veiledly. A closed walk is defined as a random walk with same start and end node. Such as a walk `1-2-3-1' is a closed walk while `1-2-3' is not. It measures an edge importance score by involving three step and four step closed walks, which is defined as:
\begin{equation}
	s_{ij} = \frac{z_{ij}^{(3)}+1}{min(k_i-1,k_j-1)} + \frac{z_{ij}^{(4)}+1}{min(k_i-1,k_j-1)}
\end{equation}
where $z_{ij}^{(k)}$ refers to the number of $k$ step closed walks that the edge $e_{ij}$ participates in. Edges are iteratively removed from the graph according to their importance score, and the remained disjoint graph components are naturally regarded as communities. 

 For other works, \cite{orecchia2014flow} is a flow based local partition method which is a refined model and theoretically proves its efficiency and effectiveness.  \cite{salnikov2016using} explains second-order Markov process in graph random walks can better reveal community structure in dynamic graphs.  NetMRF \cite{he2018network} is a Markov random field model which infers node communities by belief propagation.  \cite{ibrahim2019nonlinear} proposes a class of non-linear diffusion function between nodes, and compares their performance with non-linear transformation functions in neural networks. 

\subsection{Summary}
In this section, I explore major tracks of community detection solutions and introduce the most representative works in the recent decade. In fact, these tracks are not fully independent with each other. For example, spectral clustering and stochastic block models share similar mathematical forms. And modularity methods start to be more inveloved in deep learning methods . A graph can be viewed as either a matrix to record nodes pairwise relationship or a flow where information propagates through edges. All tracks of methods are developed from these two graph understandings. There are some other tracks of approaches as well, which are either similar to existing tracks (graph cut methods is similar to spectral clustering), or classic models waiting for more exploration (information theory models).
\section{Applications}
In this section, two types of applications are discussed from both interdisciplinary  and technical support perspectives.  For interdisciplinary supports, community detection can be applied in other domains to support better domain knowledge exploration. For technical supports, various public dataset and open-source toolkits are introduced for model comparison and evaluation. The following subsections will introduce each type of supports in detail.

\subsection{Interdisciplinary Supports}
Social media is a major relevant domain towards community detection as user connections can naturally construct social networks. Besides that, other domains such as biology, physics and neural science also involve community detection a lot to explore their object hidden connections. In the following paragraphs, social media researches are separately introduced at first, and the rest relevant researches are demonstrated together afterwards.

\subsubsection{Social Media}

\cite{papadopoulos2012community} introduces and compares a set of community detection models and provides five strategies for how to apply these models to support large scale social media analysis, including sampling techniques, local graph processing, iterative schemes, multi-level approaches and parallelization. These techniques are applied to five types of social media applications including topic detection, tag disambiguation, user profiling construction, photo clustering and event detection. 


In collaborative tagging (a.k.a. folksonomy) systems, tripartite graphs can be constructed from users, images and tags given user behaviors in the systems. By defining random walk probability values between different types of nodes, \cite{xie2014community} detects latent user communities in folksonomy graphs through a proposed Approximate Prototype Clustering (APC) method which is a similar process as K-means method. The tags of neighboring community users are ranked and selected to enrich a user’s profile. 

Due to the fact that some users create multiple sockpuppets to deceive other users or manipulate topics in online communities, \cite{kumar2017army} studies sockpuppetry to show the behavior difference between sockpuppets and ordinary users and thereafter detect sockpuppets to maintain the correct order of online communities. By analyzing 9 online communities, the study demonstrates particlar sockpuppets writing patterns (more singular first-person pronouns), write shorter sentences, and swear more) and behaving patterns (participate more controversial topics and more interact with other sockpuppets). In the end it builds up the taxonomy to differentiate and kick out sockpuppets for improving the quality of online discussion.


\cite{danescu2013no} studies individual user lifecycle evolving trend and linguistic change in online communities. In the analysis of its proposed framework, during the early stage (about one third of eventual lifespan) of individual users, they will tightly and increasing follow and be affiliated with current communities. However, after reaching to the peak point, a gap between the users'  language and community forms increases until finally they abandon the site. The main reason is because of linguistic changes. Language norms used in a community is evolving over time, and it would be harder and harder for senior users to accept this change. Once users feel themselves not able to cease the linguistic change, they will stay away from the community and eventually leave out.  Instead of analyzing individual users in online communities, \cite{kairam2012life} focuses on the lifecycle of online community itself. It finds out two types of community growth including diffusion growth,  where new members come in because of connections with existing members, and non-diffusion growth where new members come in without previous connections to the communities. It also builds up a model to predict community longivity and eventual size using graph structural features and past growth experiences. 

Users in social media such as Facebook and Twitter needs to identify their social circles either manually, or in a naive fashion by shared attributes, which is not enough to satisfy users need and lack of accuracy.  \cite{leskovec2012learning}  proposes an ego network model to define user social circles based on different aspects (i.e. family members or schoolmates) at first. Then it assigns user followings/followers into different social circles based on their profiles. It allows overlapping communities where a user can belong to multiple types social circles to current user.


Many of existing works jointly model textual content and user activities in social networks to enhance the appropriateness of detected communities. \cite{gargi2011large} presents a step-by-step approach in YouTube graphs by clustering videos into named clusters having associated tags and descriptions. It prepossess the YouTube graph and selects seed nodes using a greedy  method. After that, in order to enable community detection on millions of videos efficiently, it takes advantages of MapReduce to cluster videos in a parallel fashion. In the end, a post-processing approach is applied to refine and merge clusters by maximizing text coherence and minimizing community overlap. Having a similar goal to detect topic oriented communities, \cite{zhao2012topic} extracts social objects from emails and blogs. Based on their content, social objects are clustered into different topics. Users whose behaviors are involved in the same topic are grouped into same communities. \cite{sachan2012using} also aims to detect topically meaningful communities by considering both graph topological structure and social content in a united way. Inspired by topic modeling, it proposes a generative Bayesian model which is named as Topic User Community Model (TUCM). In the model, it assumes community membership is dependent on both topic interests from posts and graph structure. The whole learning process is guided by a Markov Chain Monte Carlo (MCMC) sampling strategy. In the end, a person can belong to multiple communities and each community can contain multiple topics. For the same purpose,  \cite{natarajan2013community} proposes an improved version of probabilistic model to jointly detect user communities and content topics by leveraging both their social connections and shared content in Twitter. In its model, community is treated as an affiliation probability distribution on users. User connections as well as associated communities topics are generated from community structure by utilizing Gibbs Sampling. Instead of using topic modeling, \cite{ozer2016community} addresses Non-negative matrix factorization method which leverages social connectivity and content information for community detection. Besides user connectivity, It considers three types of content information including word, hashtag and domain as auxiliary terms to regularize the matrix factorization process. Its empirical experiments show that word usage is the strongest indicator of user potential community affiliation among all three types of content  information.  

Mobile devices have more and more become an essential part of our daily life with the proliferation of wireless technologies. Typically, a mobile social network (MSN) is constructed by user call logs. \cite{wang2010community} introduces a Community-based Greedy algorithm (CGA) model which contains two major components including a community detection model to take care of information diffusion, and a dynamic programming model to select top $K$ most influential users given the community partition. \cite{botta2017analysis} studies on evolutionary MSN over time, reveals similar circadian and weekly patterns happened in both individual user level and community level.


\cite{traud2011comparing} studies the graphs of Facebook ``friendships'' at five U.S. universities. It investigates the community structure of each single university graph and employs visualization and quantitative analysis to measure the correlation between user communities across universities. After examining the community impact of a set of self-identified user characteristics such as residence, class year, major, and high school in all university graphs, the study concludes that student relationships are organized and dominated by multiple key factors instead of a single one. 

\subsubsection{Miscellaneous Domains}
\cite{garcia2018applications} demonstrates how modularity method detects node communities within brain graphs to reveal human neural systems functioning on the main healthy human cognition. In this study, the defined nodes in a brain graph can be flexible but need to implicate network dynamics, such as brain regions or neurons.  A d edges can reflect structural connections across spatial scales, such as bundles of axonal fibers between regions or synapses between neurons.  Similarly, \cite{liu2014network} also presents how community detection methods employed as neuroscience applications to identify the functional brain modules from multichannel and multiple subject neuroimaging data. It proposes a refined model based on modularity to detect communities in effective brain connectivity graphs. The paper quantifies brain connectivity with a defined directed information (DI) metric. It thereafter extends the Louvain method in a group of brain graphs to detect the most involved functional electrode modules in cognitive control.

Community detection techniques also contribute a lot in the biology domain to discover the hidden modules and potential bindings between proteins. \cite{he2016evolutionary} proposes a EGCPI model to identify protein complexes in the detected clusters from protein-protein interaction graphs. Based on the public Gene Ontology (GO) database, this paper constructs a protein attribute graph. With an evolutionary strategy to maximize the Independence of Cluster (IoC) fitness function, protein clusters are achieved through a genetic framework. In the end, a breadth first search (BFS) approach is applied to select sub-graphs that consist of similar proteins in each cluster based on their degree of attribute homogeneity. Having a similar propose, ClusterONE  \cite{nepusz2012detecting} is step-by-step approach to detect overlapping protein complexes from protein-protein interaction graphs. Specifically, it firstly utilizes a greedy approach to group proteins with high cohesiveness. Secondly, a merging process is taken on pairwise raw protein clusters to merge those  clusters with high predefined overlap score. At last, it leaves out those small clusters with trivial influence in graphs. \cite{lewis2010function} also reaches a conclusion that community structure in protein interaction graphs can benefit biological findings by probing different scales/resolutions in the graphs.

\cite{gupta2011evolutionary} proposes a generative model to detect communities in the DBLP dataset, which is the largest computer science bibliography database. By detecting communities in a  constructed heterogeneous bibliographical graph involving author, paper, conference and particular word/term nodes, it calculates the continue, merge and split rates of words/terms as well as authors in the graphs over time. It also shows how evolutionary communities are appeared and disappeared as well. Another work, OverCite \cite{chakraborty2013overcite}, also applies community detection in bibiliometrics, which aims to detect overlapping communities of authors, papers and venues simultaneously through the graphs constructed by these types of nodes. After detecting all node communities, the study builds up a recommendation system to use the overlapping communities as recommendation results to users.

\cite{hu2016co} proposes a CENFLD model to deal with community detection in business/enterprise graphs. A business graph involves user nodes to represent  producers, suppliers, and customers with side textual information such as recruiting messages or advertisements. To deal with business graphs' intrinsic nature of diversity, inconsistency, Implicitly and richness, the paper proposes a co-clustering factorization based approach to detect user groups. Specifically, it employs a nonnegative matrix factorization method to factorize the graph topological and textual information in individual forms (including node-feature content matrix factorization, network topology structure matrix factorization and feature-feature correlation matrix factorization). After that, it proposes a consensus principle to optimize these forms jointly. 

Besides aforementioned applications, community detection can also be applied in the chemical domain to discover force-chain clusters in granular materials \cite{bassett2015extraction}, music domain to extract musical rhythmic pattern \cite{coca2016musical}, and question-answer system \cite{fang2016community} to improve Q\&A matching accuracy.  

\subsection{Technical Supports}

After understanding how those state-of-the-art models work, a subsequent task for researchers is how to employ these models on various graphs. As most of models are too complex to be implement by individual researcher within a short period of time, open-source softwares and packages are required with an urgent need. Similarly,  benchmark graphs are also an essential part of model evaluation as they enable to testify different models under a fair condition.  In the following paragraphs, I will list and detailedly introduce several publicly available graph repositories, widely used softwares and programming toolkits. All of them significantly ease and contribute to the evaluation process by offering an environment to compare model performances under different graph scenarios.

\subsubsection{Datasets}
SNAP \cite{leskovec2015snap} is one of the most famous graph repository which contains hundreds of graph datasets collected by Stanford University\footnote{http://snap.stanford.edu/data/}. Within the graph repository, there are graphs with different semantic meanings such as social networks, citation graphs, communication graphs and web graphs. Regarding to the graph types, it includes signed graphs, temporal graphs and attribute graphs, etc. In total, the repository contains over 70 graph datasets which size range from thousands nodes to millions nodes. Many community detection papers published by Stanford University reply on this repository.

KONECT \cite{kunegis2013konect}  is an open project held by the University of Koblenz–Landau, which particularly collects large graph datasets for academic research\footnote{http://konect.uni-koblenz.de/networks/}. Up until now, there are 261 graphs with various types including (un)directed, (un)weighted, and signed graphs, etc. These graphs are also associated with different semantics such as social networks, citation graphs and communication graphs. 

LAW graph repository\footnote{http://law.di.unimi.it/datasets.php} is owned and managed by the University of Milan, which particularly focuses on big graph storage. In its repository, there are around 80 big graphs compressed via WebGraph, a graph compression framework, to  enable their quick downloading. There are various types of graphs such as web graphs, social graphs in the repository, each of which contains up to 100 million nodes and over 3 billion edges.  Even though the repository official website is still maintained by the host, after the year 2012, there is no major updates in either graph datasets or relevant published papers.

NR \cite{rossi2015network} is an interactive graph repository which enables graph downloading and interactive web analysis through their web-based platform\footnote{http://networkrepository.com/networks.php}. Currently, NR contains over 6000 graphs from 19 general domains, i.e. social netoworks, biological graphs. It covers a wide range of graph types such as bipartite graphs and temporal graphs as well. With the help of their interactive web interface, researchers can easily discover these graphs without too much effort. The lightweight platform also allows researchers to upload and share their own graphs. In this repository, the size of graph nodes is from hundreds to over 10 million. And the size of graph edges is from hundreds to over 100 million. 

There is a repository of Benchmark Graph Datasets in Github\footnote{https://github.com/shiruipan/graph\_datasets}, in which there are 31 medium-size graph datasets (each graph contains up to 10 thousand nodes). The types of graph include biological graphs, citation graphs, social networks and brain graphs. According to the repository, a series of models are assessed by graph classification and community detection evaluation tasks using these graph datasets.

\subsubsection{Softwares}

Pajek\footnote{http://mrvar.fdv.uni-lj.si/pajek/} is a public network analysis  toolkit particularly for analysis and visualization on very large graphs. It is a well-maintained nonprofit project and the latest version is release in September, 2019. This tool offers hands-on manuals in all major languages including English, Chinese and Spanish. Originally, this software is deigned to run on Windows system. However, in recent years, several extensions are built up to enable running Pajek in both Mac and Linux system as well. Pajek is one of the best known graph analysis toolkit. The eligible graph format created by Pajek even becomes a standard format of graph data. Besides visualizing graphs into a 2-D plot, it can also calculate several major graph metrics such as triangle numbers and graph modularity score.  Several basic community detection, such as Louvian method, are also embedded in Pajek to reveal high order graph structure.

NetMiner\footnote{http://www.netminer.com/main/main-read.do} is a commercial software for exploratory analysis and visualization on graphs. It is owned by a Korean company named CYRAM. The software is similar to Pajek software but has more interactive functions. As one of its advantages, It supports large network analysis on graphs with thousands of nodes, which can satisfy most of needs in the social science domain. Moreover, five community detection methods, including Modularity method and label propagation method, are implemented in the software to support more intricate graph analysis.  

CFinder\footnote{http://www.cfinder.org/} is a free graphic tool to find graph communities and visualize the generated node cliques in 2-D plots. It is developed based on Java Spring framework and has a similar layout to other previously mentioned softwares. It requires to install Java Runtime Environment (JRE) as a prerequisite. CFinder is able to run under all main platforms (i.e. Windowns, Mac and Linux ) and can be easily installed by researchers with no technical background. The latest version is released in the year 2014 and the latest paper is published in 2016. To summary, it is a possible software option for researchers to explore and visualize graph communities but the back-end support is a little bit out of date.  

Gephi\footnote{https://gephi.org/} is a leading visualization and exploration tool for social networks. The goal of Gephi is to make better analysis to find patterns to uncover graph structures and visualize the result in an intuitive way, which makes it easier for users to understand. In its interface, there is a visualization window along with a bunch of functional buttons. By clicking on the buttons or change the values of the related boxes, it can directly change the visualization layout. Even though Gephi offers a Python library as well, its software is still more popular and better accepted by users. One advantage of the software is that dynamic networks can be also conveniently explored. Another advanced feature is that Gephi is able to render 3-D plots in the visualization window. The recent major update is in 2017, which is able to support all types of platforms including Windows, Mac and Linux.

UCINET\footnote{https://sites.google.com/site/ucinetsoftware/home} is a software funded by a start-up company named Analytic Technologies. It is particularly designed for Windows system and needs to be purchased after 90 days free trial. This software is pretty active and the latest version is release in March 2020. Although the official website claims the software can handle graphs with up to 30 thousand nodes. However, empirical experiences show that its process will run slow when the graph size is over five thousand nodes. 

GUESS\footnote{http://graphexploration.cond.org/} is an exploratory data analysis and visualization tool for graphs. Claimed in its official website, the software contains a  script language called Gython to handle large graph visualization via Java Applet. There are also several community detection methods embedded in the software, which can help to visualize graph hidden structure. However, as there is no major update since 2007, researchers have abandoned this software. Even though,  it still offer some  insights for graph analysis. Unlike other softwares only need users to click buttons in the interactive interface,  GUESS requires users  to type commands instead for generating the graph plots. This feature raises the bar to learn this software. 

ORA-LITE\footnote{http://www.casos.cs.cmu.edu/projects/ora/} is a tookit for dynamic graph assessment, which is developed by the Carnegie Mellon University . It is a software mainly used for dynamic networks, which means the graphs they analyze change over time. As it claims in the official website, this software can not only find out the community structure of  dynamic graphs but also other graph metrics. Although  the time complexity is usually high in dynamic network analysis, this software is able to handle graphs with millions nodes. Another advantage of this software is that it offers multi-language tutorials and even holds Google groups for users to communicate. The latest version of this software is published in January 2020. But it only support to run on Windows system.

Cytoscape\footnote{http://www.cytoscape.org/} is an open-source platform originally designed for biological researches. Through the development in recent years, it currently turns to be a general platform for all types of graph analysis and visualization. Cytoscape offers a Javascript API to allow rendering graph plots in web applications and a Java API to allow connecting to remote servers.  Inside its software, there are a set of basic features installed by default. Beyond that, users have a choice to enable advanced features by installing plugins by themselves. The latest version of this software is released in June 2019. And based on the announcement from its official website, the number of software daily downloads is huge and its discussion forum is pretty active until now.

MuxViz\footnote{http://muxviz.net/index.php} is a framework for the multilayer graph analysis and visualization. It allows interactive visualization and exploration  for graphs with multi-type relationships and attributes, which is the most outstanding feature of this software. MuxViz is developed based on R and GNU Octave, which means it requires to install R in advance and can only deal with at most middle-size graphs. As a open-source software, it can run on Windows, Linux and Mac OS X. However, its latest release is in 2015, which is a bit out of date. 

Visone\footnote{https://visone.info/} is a free toolkit for analyzing and visualizing social networks, which is a long term project managed by the University of Konstanz. It can run either from terminal or through its interactive interface. The software supports all types of platforms including Windows, Mac and Linux. As It is originally written in Java, the current version released in 2019 requires at least Java 8 to be installed in advance. Many community detection methods are embedded in it such as Modularity and spectral methods. Another feature is that the rendered plots can be directly exported and saved to local disk. 

\subsubsection{Programming Toolkits}

As one of the most popular programming language in statistics, R contains many packages which are particularly designed for graph analysis and community detection. Wrapped as a high level programming language, R is not able to handle large scale graphs as other languages such as Java and C++. However thanks to its simplified design, it is able to support more complex models than other languages. In the following paragraphs, I will briefly cover three widely used packages including modMax, networktools and RANN.

modMax\footnote{https://cran.r-project.org/web/packages/modMax/modMax.pdf} is a R packages which contains a lot of Modularity based methods. As one of the largest methodological track in community detection, Modularity has a lot of variants and optimization solutions. In modMax, many variants are covered such as Modularity optimization using fast greedy, simulated annealing, spectral and  genetic methods.

networktools\footnote{https://cran.r-project.org/web/packages/networktools/networktools.pdf} is another newly developed R package particularly for graph analysis. Its functions include not only basic community detection algorithms but also have visualization functions. 

RANN\footnote{ https://github.com/jefferis/RANN} provides a set of fast $k$ nearest neighbor search models. It is a R wrapper of ANN library, which is originally written in C++.

igraph\footnote{https://igraph.org/} is one of the most widely used packages for community detection. It is originally written in C++ but offers R, Python and Mathematica wrappers. igraph package is frequently updated and the recent release is released in March 2020. There are a bunch of widely used community detection models implemented in the package such as Modularity, Walktrap, Label Propagation and Infomap method.


NetworkX\footnote{https://networkx.github.io/} is a fundamental Python package which defines the basic data structures used for graph analysis. Most of Python packages related to graph analysis are inherited from NetworkX. Within the package, it also contains several components such as centrality, clique, clustering in which several functions are related to community detection models or graph evaluation metrics.

SNAP\footnote{http://snap.stanford.edu/index.html} is a general graph mining library. It is written in C++ and easily scales to massive networks with hundreds of millions of nodes. The package has both C++ version and a Python wrapper. So far, it is the quickest package for community detection in large scale graphs. Another advantage of this package is that its embedded models keep updating frequently. It contains tens of state-of-the-art models for overlapping community, dynamic graph community, and Modularity based community detection. 

CDlib\footnote{https://cdlib.readthedocs.io/}  is a newly announced Python library for complex network analysis. The recent version is released in February 2020. This software is extended from NetworkX but offers more community detection methods and  evaluation metrics. Overall, It contains over 40 community detection methods about node clustering, edge clustering and overlapping community detection. Besides, it also offers functions to calculate more than 20 widely used evaluation metrics.

\subsection{Summary}
In this section, I firstly introduce how community detection can support researches in other domains such as citation recommendation, biology exploration and social media analysis. After that, a bunch of public resources, such as datasets and toolkits, are introduced to support community detection model evaluation. Softwares are easier to learn but programming toolkits have more flexibility to deal with complicated scenarios. Here are some tips and hints about how to choose toolkit if needed: 

First, toolkit selection is purpose oriented.  Researchers need to clarify their research tasks, and thereafter to select the proper toolkit. For example, if researchers need to analyse dynamic graphs, 
ORA-LITE is definitely the first choice to consider. 

Second, identify the preference of flexibility or simplicity. If researchers are lack of technical background or don't want to spend too much time for learning, then softwares should be a better choice than programming toolkits. While if researchers want more customized functions, programming toolkits will be a better choice. 

Third, price matters as well. Some paid softwares may contain more features than free ones. Researchers have to make sure whether it is worthy to invest on these commercial softwares such as UCINET.

Fourth, keep updated. There are many toolkit came out each year. Researchers need to keep up with the latest toolkit and explore the possibility to apply it into personal researches.  Usually toolkits released by high-tech companies and top tier universities are associated with enriched tutorials. They are easier for new users to start learning and more reliable to use compared with other open source packages shared via Github repository.

Overall, researchers need to consider all above mentioned tips before they choose datasets and softwares to use in their academic researches. The appropriate selection may significantly alleviate their workloads in model performance analysis.



\section{Evaluation}
Once the community partition result is retrieved by taking a certain type of model, a subsequent approach is to evaluate model performance and run comparisons among different models. Therefore, how to carry out comparative analysis and what metrics should be reported as benchmarks are two imperative factors to be considered. In the following sections, I will address these two questions separately by introducing several highly cited papers about comparative analysis and a number of best known metrics widely used in industry and academia.    

\subsection{Comparative Studies}
\cite{yang2015defining} compares 13 community scoring functions and employ them on 230 graph datasets with various topics (social, collaboration, and purchasing graphs, etc.) and different sizes (from hundreds of thousand to hundreds of millions of nodes and edges in graphs) to generate communities. The 13 scoring functions lie in four categories including Internal Connectivity, External Connectivity, the combination of Internal \& External Connectivity and Network Model (Modularity). To assess these scoring functions' community adequacy, four metrics are thereafter to considered, including separability, density, cohesiveness and clustering coefficient (all these metrics will be explained in the later section). There are significant performance differences among all scoring functions  where Conductance (it measures the ratio of edges linking outside of the community) and Triad participation ratio (the ratio of nodes in the same cluster forms a triad) achieves the best performance, while Modularity score is with poor performance.

\cite{orman2012comparative} argues that current community detection methods totally ignore the topological nature
of the communities as two methods may have similar evaluation performance but totally different community distributions. It claims that adding community topological metrics such as community-wise density, average distance, internal transitivity, and hub dominance may give more supportive understandings about the community partition result. Moreover, by testing on several real-world graph datasets as well as synthetic graphs using different types of approaches (e.g., Modularity-based and diffusion-based approaches), it also empirically shows that there is no clear correlation between the method performance metrics (e.g., rand index and normalized mutual information) and community topological metrics. 

\cite{hric2014community} questions the  appropriateness to simply use the node metadata information (e.g., node category or membership) as the criteria to build up the ground truth. By running 11 different community detection models on 16 different graph datasets, it claims that there is a trivial correlation between the communities constructed by graph metadata and model-detected communities. Thus, it concludes that metadata groups are not able to fully infer graph topological structure as they are separate views for the graphs. Instead, graph metadata can be taken as either auxiliary information to enhance community partition performance or another perspective to interpret graphs.  

By employing 8 different models on Lancichinetti-Fortunato-Radicchi (LFR) benchmark graph, \cite{yang2016comparative} summarizes how well these models can handle graph scale, community size recovery problems. All model performance and computing time are also assessed and compared with each other. In the end, the paper offers some insights about how to choose the proper model given graph descriptive parameters like LFR mixing parameter.

\cite{abrahao2012separability} analyzes communities by taking account of a broadspectrum of structural properties. The analysis reveals nu-ances of the structure of real and extracted communities

We extract different classes of communities thatcan be grouped into two categories: intrinsically-defined andextrinsically-defined communities.

a large set of structural properties and ten differentcommunity detection procedures to produce examples of dif-ferent structural classes.


\cite{wang2015community}
\subsection{Metrics} 
\cite{chakraborty2017metrics}(survey),\cite{murray2012using},\cite{chakraborty2014permanence},\cite{li2015measuring},\cite{shen2010spectral},\cite{aldecoa2013surprise},\cite{hu2010measuring},\cite{aldecoa2012closed},\cite{zhang2015evaluating},\cite{nikolaev2015efficient},



In this chapter we discuss how to evaluate the generated community quality, which is also called community evaluation \cite{chakraborty2017metrics}. Typically, a network can be divided into several subgraphs with explicit or implicit relationships. Users need to decide whether the generated community is of good quality and is the most appropriate as a clustering result. Typically, there are some widely accepted metrics such as clustering accuracy are used for community evaluation. \cite{lancichinetti2008benchmark} offers a bunch of graph algorithms as well as metrics to be used ad benchmark in community detection model comparison. However, in some cases, not all widely accepted metrics are equally important to evaluate the community quality because  the research questions are different \cite{fortunato2010community}. For example, to solve the dynamic network problems, time related indicator should be more important for model assessment. And for solving large scale network community problems, the performance of an algorithm in terms of time complexity is also a crucial factor.

Since a bunch of community algorithms have been proposed for both overlapping and non-overlapping networks, indicators used to solve the community detection problems on two different  types of graphs are calculated in the different ways.  In other words, we should separately discuss the metrics used in terms of different types of networks. In most of the cases, we have to know the real community structure (ground truth) for tested dataset, so that we can compare our model results with other baseline methods, which is the widely applied approach for model evaluation.  

In this chapter, although there are many taxonomies for metric categorization, such as metrics for undirected or directed networks, we classify and report the evaluation metrics as overlapping metrics and non-overlapping metrics \cite{fortunato2010community,schaeffer2007graph}. While some of the metrics are common in the two scenarios such as the number of triangles in community \cite{leicht2008community,arenas2007size}, we still think it is the best way to separate metrics in those categories.

In the following two sections, we discuss the metrics used to measure the similarity between the detected and the ground-truth community structures for either non-overlapping or overlapping community detection approaches.  In fact, most of existing communities focus on non-overlapping community detection, as it is a simpler scenario, some studies do point out the significance of overlapping community evaluation, mostly in a series of papers published in SNAP \footnote{http://snap.stanford.edu/index.html}. Within those papers, many overlapping metrics are introduced.


\subsubsection{Non-overlapping Metrics}
In this section, we discuss the metrics used to measure the similarity between the detected and the ground-truth community structures for non-overlapping community structure.

We define, given a graph $G(V,E)$,  $\Omega = \{w_{1},w_{2},...w_{k}\}$ is the set of the detected communities users generated with some non-overlapping community detection models. And $C = \{c_{1},c_{2},...,c_{k}\}$ is the set of the ground truth communities given in advance. $N = |V| = \sum_{k \in K}|w_{k}|= \sum_{j \in J}|c_{j}|$  is the total number of the nodes in the original graph. $|N_{c_{j}}| = |c_{j}|$ and $|N_{w_{i},c_{j}}| = |w_{i} \cap c_{j}|$ refers to the number of nodes in each community as well as the number of common nodes detected in the two communities from both graph partition $C$ and $\Omega$.

After we define some basic symbols to represent the graph community detection results, we can use the following evaluation metrics to judge whether our predicted community partition is good or bad to estimate the ground truth community results. The whole introduced evaluation metrics for non-overlapping community detection can be seen from Table \ref{tab:non-overlapping-metric}. In the following paragraphs, we start to introduce each metric one by one.

\begin{table}
	% \scriptsize
	\centering
	%   \vspace{-3em} 
	% \renewcommand{\tabcolsep}{2pt}
	\begin{tabular}{|l|l|} \hline
		\textbf{Metric} &  \textbf{Expression}    \\ \hline
		
		Purity& $\frac{1}{N}$$\sum_{k} \argmax_{j} |w_{k},c_{j}|$  \\ \hline 
		F-measure & $\frac{2*Purity(\Omega,C)*Purity(C,\Omega)}{*Purity(\Omega,C)+Purity(C,\Omega)}$ \\ \hline 
		Rand index& $\frac{TP+TN}{TP+FP+FN+TN}$ \\ \hline 
		$F_{\beta}$, precision P, recall R& $\frac{(\beta^{2}+1)PR}{\beta^{2}(P+R)} $; $P = \frac{TP}{TP+FP}$;$R = \frac{TP}{TP+FN}$ \\ \hline  
		Adjusted rand index & $\frac{\sum_{ij}\binom{N_{w_{i}c_{j}}}{2} -\sum_{i}\binom{N_{w_{i}}}{2} \sum_{j}\binom{N_{c_{j}}}{2}/\binom{N}{2}}{1/2*\sum_{i}\binom{N_{w_{i}}}{2}+\sum_{j}\binom{N_{c_{j}}}{2}--\sum_{i}\binom{N_{w_{i}}}{2} \sum_{j}\binom{N_{c_{j}}}{2}} $  \\ \hline 
		Entropy& $H(\Omega) = - \sum_{k} \frac{|w_{k}|}{N}log \frac{w_{k}}{N}$ \\ \hline 
		Normalized mutual information & $ \frac{I(\Omega,C)}{[H(\Omega)+H(C)]/2}$ ; $I(\Omega,C) = \sum_{k}\sum_{j}\frac{|w_{k}| \cap c_{j}}{N} log \frac{N|w_{k} \cap c_{j}|}{|w_{k}||c_{j}|}$ \\ \hline 
		Variation of information& $ \sum_{i,j} r_{i,j}[log(r_{ij}/p_{i}) + log(r_{ij}/q_{j})]$= $H(\Omega) + H(C)-2I(\Omega,C)$ \\ \hline 
		Modified purity& $\sum_{i}\sum_{u \in w_{i}} \frac{w_{u}}{w}Purity(u,\Omega,C) $\\ \hline 
		Modified ARI& $\frac{\sum_{ij}W(w_{i} \cap W(c_{j})) - \sum_{j}W(w_{i})W(c_{j})/W(V)}{\frac{1}{2}(\sum_{i}W(w_{i}+\sum_{j}W(c_{j}))) - \sum_{i}W(w_{i})\sum_{j}W(W(c_{j}))/W(s)}$\\ \hline 
	\end{tabular}
	\caption{Evaluation metrics for non-overlapping community detection results}
	\label{tab:non-overlapping-metric}
	
\end{table} 

Separability measures the ratio between the internal and the external number of edges of  node set $S$: $g(S) = \frac{m_S}{c_S}$

Density builds on intuition that good communities are well connected. One way to capture this is to characterize the fraction of the edges (out of all possible edges) that appear between the nodes in $S$, $g_S = \frac{m_S}{n_S(n_S-1)/2}$.

Clustering coefficient is based on the premise that network communities are manifestations of locally inhomogeneous distributions of edges, because pairs of nodes with common neighbors are more likely to be connected with each other.

Cohesiveness characterizes the internal structure of the community. Intuitively, a good community should be internally well and evenly connected, i.e., it should be relatively hard to split a community into two sub communities. We characterize this by the conductance of the internal cut. Formally,$g(S)=max_{S^{'} \in S} \phi(S)$ where $\phi(S^{'})$ is the conductance of $S^{'}$measured in the induced subgraph by S. Intuitively,conductance measures the ratio of the edges in $S^{'}$ that point outside the set and the edges inside the set $S^{'}$. A good community should have high cohesiveness (high internal conductance) as it should require deleting many edges before the community would be internally split into disconnected components 

\textbf{Purity}

Purity \cite{schutze2008introduction,lin2005foundations} is introduced where each detected community
is assigned to the ground-truth label which is most frequent in the community. It is calculated by the ratio of nodes that are correctly assigned labels in the community. The formula to calculate it can be defined as:

$$
Purity(\Omega,C) = \frac{1}{N} \sum_{k} max|w_{k},c_{j}|
$$
The purity score is in range from 0 -1. 1 means the detected community is perfectly matched with the ground truth label. And the 0 means the opposite. From the definition, it is easy to find that the fomula is not symetric. Which means that $Purity(\Omega,C)$ and $Purity(C,\Omega)$ is different.   In practice, $Purity(\Omega,C)$ is usually used and called as the simple purity. While the $Purity(C,\Omega)$ is called the reverse purity  \cite{artiles2007semeval,danon2005comparing}. As the ground truth label is always known in advance for evaluation, hence the previous one makes more sense as we use the known labels to assign for predicted communities. However, purity is really sensitive to the number of communities for the original graph. If the number of communities increases, the purity score is more likely to be higher. The extreme case is that purity will be 1 if every node is assigned to the community itself only. While the case of inverse purity is the opposite. The most extreme case happens when all the nodes are assigned to the same community. Hence, purity iteself is not an ideal evaluation metric to judge whether the community partition is good or bad, especially when the number of clusters/ communities is unknown or undecided.

\textbf{F-Measure}

To solve this problem, Newman \cite{newman2004finding} introduced an additional constraint generally adopted in cluster
analysis rather consists in processing the F-Measure, which is the harmonic mean of
both the versions of the purity.

$$
F = \frac{2*Purity(\Omega,C)*Purity(C,\Omega)}{*Purity(\Omega,C)+Purity(C,\Omega)}
$$

This formula considers both of the purity and the inverse purity, which can balance the two metrics at one time.

\textbf{Rand index}

We can also regard the community detection problem as a classification problem \cite{hubert1985comparing}.
A true positive (TP) decision assigns two same-labeled nodes to the same community;
a true negative (TN) decision assigns two different labeled nodes to different communities. There are two types of errors we can commit. A (FP) decision assigns two different labeled nodes to the same community. A (FN) decision assigns two same labeled nodes to different communities. Hence, based on the definition of the clustering analysis, we can still use the rand index calculated for evaluating clustering result:

$$
RI = \frac{TP+TN}{TP+FP+FN+TN}
$$

\textbf{Precision, Recall, $F_{\beta}$}

Derived from the methods mentioned above, we suddenly find some basic classification evaluation metrics can be natually used for evaluating community detection result. Hence, we can still use precision, recall and $F_{\beta}$. As RI gives equal wieght to FP and FN based on its definition, we can use the $F_{\beta}$ to measure the same metrics in a more flexible way by involving one more parameter $\beta$ in the formula. We can use the $F_{\beta}$  to penalize FNs more
strongly than FPs by selecting a value $\beta >$  1. Thus, the defined precision, recall and $F_{\beta}$ can be calculated as:

$$
\begin{aligned}
& F_{\beta} = \frac{(\beta^{2}+1)PR}{\beta^{2}(P+R)} \\
& P = \frac{TP}{TP+FP}\\
&R = \frac{TP}{TP+FN}
\end{aligned}
$$ 

\textbf{Adjusted  Rand Index}

Based on the previous discussion, we can see that simple rand index is not able to capture enough community detection information and get a relevant reliable feedbacks for the partition result. Hence, Adjusted Rand Index (ARI) is also raised \cite{hubert1985comparing}. It is less sensitive to the number of communities \cite{nguyen2009information}. The chance correction is based on the general formula defined for any measure M, which is a typical standardization function widely used in current days. 

$$
M_{c} = \frac{M-E(M)}{M_{max}-E(M)}
$$
where $M_{c}$ is the chance-corrected measure, $M_{max}$ is the maximal value that M can
reach, and E(M) is the value expected for some null model. Under the assumption that the partitions are generated randomly given a community number, the original function of rand index can be standardized by using the function mentioned above as:

$$
\begin{aligned} 
%&RI =  \frac{\sum_{ij}\binom{N_{w_{i}c_{j}}}{2}}{1/2*[\sum_{i}\binom{N_{w_{i}}}{2}+\sum_{j}\binom{N_{c_{j}}}{2}]}  \\
&ARI = \frac{\sum_{ij}\binom{N_{w_{i}c_{j}}}{2} -\sum_{i}\binom{N_{w_{i}}}{2} \sum_{j}\binom{N_{c_{j}}}{2}/\binom{N}{2}}{1/2*[\sum_{i}\binom{N_{w_{i}}}{2}+\sum_{j}\binom{N_{c_{j}}}{2}]-\sum_{i}\binom{N_{w_{i}}}{2} \sum_{j}\binom{N_{c_{j}}}{2}/\binom{N}{2}} 
\end{aligned}
$$

This metric has an upper bound of 1. And if the value is below 0, it means that the partition result is even worse than a randomly generated community partition. 


\textbf{Entropy}

Entropy probabily is the most important metric to evaluate whether the prediction result is good or bad in machine learning tasks. Community detection also borrows idea from entropy to judge the partition. Lower entropy score refers to a more stable status. And the definition of it is shown as:

$$H(\Omega) = - \sum_{k} \frac{|w_{k}|}{N}log \frac{w_{k}}{N}$$

\textbf{Normalized Mutual Information and Mutual Information}

NMI \cite{schutze2008introduction,strehl2002cluster,fortunato2009community} is another quite popular evaluation metric in community detection. And it is defined as 
$$NMI =  \frac{I(\Omega,C)}{[H(\Omega)+H(C)]/2}$$

While $I(\Omega,C)$ refers to the mutual information of the predicted community result with the ground truth result. It calculates how much similar the two results look like. The calculation function is the same as the normal MI function used in traditional machine learning. NMI is always a number between 0 and 1. A major problem of NMI is that it is not
a true metric, i.e., it does not follow triangle-inequality thereon. MI function is calculated as below:

$$I(\Omega,C) = \sum_{k}\sum_{j}\frac{|w_{k} \cap c_{j}}{N}| log \frac{N|w_{k} \cap c_{j}|}{|w_{k}||c_{j}|}$$


\textbf{Variation of information}

NMI has a very serious drawback called selection bias problem. The value is biased to the number of generated communities. Larger community size pretend to obtain a higher NMI score. To overcome the drawbacks of NMI, Variation of Information (VI) \cite{meilua2007comparing,kraskov2005hierarchical} is defined which is able to calculate the shared information distance obeys the triangle inequality. It is defined as:

$$ VOI = \sum_{i,j} r_{i,j}[log(r_{ij}/p_{i}) + log(r_{ij}/q_{j})] $$ 
or
$$VOI =  H(\Omega) + H(C)-2I(\Omega,C)$$
where $p_{i} = \frac{w_{i}}{N}$, $q_{j} = \frac{c_{i}}{N}$ and $r_{ij} = \frac{w_{i} \cap c_{j}}{N}$. 


\textbf{Modified purity}

\cite{orman2012comparative,labatut2015generalised} argue that the traditional metrics consider a community structure and ignore the network topology. They add the weights information of a node to calculate purity score. First, as the same Purity function mentioned above, they represent the function use another term:

$$
Purity(u,\Omega,C) = \delta(C_{j}|s.t. N_{w_{i}c_{j}} \text{ is maximum})
$$

After they add link weights to the original purity function, it terms to be:
$$
MP = \sum_{i}\sum_{u \in w_{i}} \frac{w_{u}}{w}Purity(u,\Omega,C) 
$$

where $W=\sum_{v}w_{v}$ is the sum of all the weights of the nodes in a community or in the whole graph. This function can keep the modified purity result in a range of 0 to 1.

\textbf{Modified ARI}

Taking the similar idea from modified purity function, since Rand Index is based on pairwise comparisons, it is not possible to
isolate the individual effect of each node, previous studies propose to use the product of the two
corresponding nodal weights: $w_{u}w_{v}$. Then for any subset of nodes S, it can be translated
as: $W(S) = \sum_{u,v \in S} w_{v}w_{u}$. And the modified ARI turns into:

$$ARI = \frac{\sum_{ij}W(w_{i} \cap W(c_{j})) - \sum_{j}W(w_{i})W(c_{j})/W(V)}{\frac{1}{2}(\sum_{i}W(w_{i}+\sum_{j}W(c_{j}))) - \sum_{i}W(w_{i})\sum_{j}W(W(c_{j}))/W(s)}$$
\subsubsection{Summary}
In this section, we introduce twelve basic evaluation metrics in non-overlapping community detection methods.  Those metrics are almost all metrics used among current researches to evaluate the non-overlapping community quality. Some of the metrics are previously raised, which were classic metrics used for years, such as Mutual information and Rand index. However, in recent years, researchers argue those metrics do not consider the normalization issue, which causes bias towards the evaluation. Therefore, many refined metrics such Adjusted rand index and Normalized mutual information start to be used more often. Based on experience, we suggest to use the normalized metrics for model evaluation, which reduces the negative effect on community numbers or biased  community size distribution. Precision, Recall and F measures are the three basic metrics for non-overlapping community detection, we suggest to at least report F measure for academic papers. As for other metrics, they can be chosen based on the evaluation task requirements.

\subsubsection{Overlapping Metrics}
In overlapping community detection evaluation, the same as the non-overlapping community detection,
We define, given a graph $G(V,E)$,  $\Psi = \{\psi_{1},\psi_{2},...\psi_{k}\}$ is the set of the detected communities generated with some community detection models. And $C = \{c_{1},c_{2},...,c_{j}\}$ is the set of the ground truth communities given in advance. $N = |V| = \sum_{\psi \in \Psi}|\psi_{k}|= \sum_{j \in J}|c_{j}|$  is the total number of the nodes in the original graph where $|N_{c_{j}}| = |c_{j}|$ and $|N_{\psi_{i},c_{j}}| = |\psi_{i} \cap c_{j}|$. Hereby, we will introduce four kinds of basic metrics widely used in overlapping community detection. We can see that the metrics are sort of derived from non-overlapping community detection and there are only small changes on the non-overlapping community detection metrics one. The calculation function of the four metrics in shown in Table \ref{tab:overlapping-metric}.
\begin{table}
	% \scriptsize
	\centering
	%   \vspace{-3em} 
	% \renewcommand{\tabcolsep}{2pt}
	\begin{tabular}{|p{8cm}|p{6cm}|} \hline
		\textbf{Metric} &  \textbf{Expression}    \\ \hline
		
		
		Overlapping normalized mutual information &$ONMI(X|Y) = 1 - \frac{H(X|Y)+H(Y|X)}{2}$  \\ \hline 
		Omega index & $\frac{Omega_{u}(\Psi,C)-Omega_{e}(\Psi,C)}{1-Omega_{e}(\Phi,C)}$ $Omega_u = \frac{1}{M}\sum_{j=1}\argmax (|\Psi|,|C|)|t_{j}(\psi_{i})\cap t_{j}(c_{j})|$ $Omega_e = \frac{1}{M^2}\sum_{j=1}\argmax (|\Psi|,|C|)|t_{j}(\psi_{i})\cdot t_{j}(c_{j})|$ \\ \hline 
		Generalized external index& $a_{G}(i,j) =min \{\alpha_{\Psi}(i,j),\alpha_{C}(i,j)\} + min \{\beta_{\Psi}(i),\beta_{C}{i}\}+ min \{\beta_{\Psi}(j),\beta_{C}{j}\}$ $a_{G}(i,j) =abs |\alpha_{\Psi}(i,j)-\alpha_{C}(i,j)| + abs|\beta_{\Psi}(i)-\beta_{C}{i}|+ abs|\beta_{\Psi}(j)-\beta_{C}{j}|$ $GEI(\Psi,C)=\frac{a){G}}{a_{G}+d_{G}}$ \\ \hline 
		F1-score& $\frac{1}{2} (\frac{1}{|\Psi|}\sum_{\psi_{i} \in \Psi}F1(\psi_{i},C_{g_{i}})+\frac{1}{|C|}\sum_{c_{i} \in C}F1(\psi_{g^{\prime}(i)},C_{i}))$  \\ \hline  
	\end{tabular}
	\caption{Evaluation metrics for overlapping community detection results}
	\label{tab:overlapping-metric}
	
\end{table} 

The metrics include Overlapping normalized mutual information, Omega index ,  Generalized external index and F1-score. As precision, recall and F1-score  are calculated in the same track, the difference of F1-score between overlapping and non-overlapping community detection is the same as the difference between precision and recall in the two community scenarios. Although it seems that the calculation is pretty complex, after the introductions listed in the following paragraphs, we can see that all metrics are of pretty naive and straight forward thoughts on community evaluation. In the following paragraphs, we will introduce them one by one.

\textbf{Overlapping normalized   mutual information}

The conditional entropy of a cluster $\psi_{k}$ given $C_{l}$ is
defined as $H(\psi_{k}|C_{l}) = H(\psi_{k},C_{l}) - H(C_{l})$. The entropy of $\psi_{k}$ with respect to the entire vector $C$ is based on the best matching between $\psi_{k}$ and any component of $C$ given by
vector $C$ is based on the best matching between $\psi_{k}$ and any component of $C$ given by:

$$
H(\psi_{k}|C) = min_l (\psi_{k}|C_{l})
$$

And the normalized conditional entropy of$\Psi$ to $C$ is naturally calculated as:
$$
H(\Psi|C) = \frac{1}{|C|}\sum_{k} \frac{H(\psi_{k}|C)}{H(\Psi_{k}) }
$$
Similarly, we can define $H(C|\Psi)$. And the overlapping normalized mutual information (ONMI) \cite{mcdaid2011normalized} is defined as: 
$$ONMI(\Psi|C) = 1 - \frac{H(\Psi|C)+H(C|\Psi)}{2}$$

If the overlapping part is 0, this function is also able to represent non-overlapping NMI score. In ONMI, it calculated the conditional entropy from both sides ($\Psi \rightarrow	 C$ and $C \rightarrow	\Psi$). Higher ONMI score refers to the higher correlation between ground truth communities and generated communities.

\textbf{Omega index}

Omega index \cite{murray2012using} can be regarded as the overlapping version of the Adjusted Rand Index \cite{collins1988omega,murray2012using}, which is another critical indicator to measure the overlapping community detection performance. Omega index is defined in the following way \cite{gregory2011fuzzy,havemann2011identification}:
$$\Omega = \frac{Omega_{u}(\Psi,C)-Omega_{e}(\Psi,C)}{1-Omega_{e}(\Phi,C)}$$

The unadjusted Omega index $Omega_u$ is defined as:
$$Omega_u = \frac{1}{M}\sum_{j=1}\argmax (|\Psi|,|C|)|t_{j}(\psi_{i})\cap t_{j}(c_{j})|$$
And the expected Omega index in the null model $Omega_e$
is given by,
$$Omega_e = \frac{1}{M^2}\sum_{j=1}\argmax (|\Psi|,|C|)|t_{j}(\psi_{i})\cdot t_{j}(c_{j})|$$

Omega index refers to how much the generated community has overlapped components with ground truth over than the null model (random overlapping communities). 

\textbf{Generalized external index}

\cite{campello2007fuzzy,campello2010generalized} introduce another evaluation metric called Generalized external index (GEI) for comparing the predicted community and ground truth community result. It includes the following information:
\begin{itemize}
	\item  $\alpha_{\Psi}(i,j)$: Number of communities shared by nodes $i$ and $j$ in partition $\Psi$	
	\item   $\alpha_{C}(i,j)$: Number of communities shared by nodes $i$ and $j$ in partition $C$.
	\item  $\beta_{\Psi}(i,j)$: Number of communities to which node $i$ belongs in $\Psi$ minus 1.
	\item $\beta_{C}(i,j)$: Number of communities to which node $i$ belongs in C, minus 1.
\end{itemize}

Based on the definitions above, the agreements $a_{G}$ and disagreements $d_{G}$ associated  are defined as: 
$$
\begin{aligned}
&a_{G}(i,j) =min \{\alpha_{\Psi}(i,j),\alpha_{C}(i,j)\} + min \{\beta_{\Psi}(i),\beta_{C}{i}\}+ min \{\beta_{\Psi}(j),\beta_{C}{j}\} \\
&d_{G}(i,j) =abs |\alpha_{\Psi}(i,j)-\alpha_{C}(i,j)| + abs|\beta_{\Psi}(i)-\beta_{C}{i}|+ abs|\beta_{\Psi}(j)-\beta_{C}{j}|
\end{aligned} 
$$ 

And the final GEI formula is:

$$GEI(\Psi,C)=\frac{a_{G}}{a_{G}+d_{G}}$$

Instead of calculation the community similarity on community level, GEI evaluate the performance of generated communities on pairwise node level. From the extremely complicated formulas mentioned above, we can see the fundamental idea is to check whether their relationships keep the same on each pair of nodes in the graphs. While the calculation takes much more time complexity then the rest indicators mentioned above.

\textbf{F1-score}

\cite{yang2013overlapping} uses average F1-score to measure the equivalence of two overlapping partitions. It is defined to be the average of the F1-score of the best matching ground-truth community to each detected community. The calculation formula is defined as:

$$F1 = \frac{1}{2} (\frac{1}{|\Psi|}\sum_{\psi_{i} \in \Psi}F1(\psi_{i},C_{g_{i}})+\frac{1}{|C|}\sum_{c_{i} \in C}F1(\psi_{g^{\prime}(i)},C_{i}))$$


\subsubsection{Summary}

In this section, we introduce four different metrics for overlapping community evaluation. We can see there are some connections on those metrics with non-overlapping community detection evaluation metrics but with much more complex considerations. They either evaluate from the community level similarity or pairwise node level similarity, which offers more potential tracks for overlapping community detection. However, as overlapping community is only a part of the whole community detection family, more future works need to work on both model side and evaluation side.
\subsection{Summary}


\section{Future Work}
Looking ahead, my next steps are to keep track of state-of-the-art activities in the community detection domain. With respect to the personalized community detection task, as mentioned above, I plan to design my own method for offline community tree construction and node representation, preferably by having the entire offline process developed end-to-end without arbitrary parameters and methods selection. For the cross-graph community detection task, to avoid an extra step to detect communities, I want to launch the community detection process into the main pipeline to achieve a real end-to-end model, as well as include more textual information from users in the model in order to construct attribute graphs. For the community-aware dynamic product summarization, a further step will be to integrate community detection into the joint training process. In addition more advanced text generative models need be utilized, such as Pointer Network \cite{vinyals2015pointer} and Copynet \cite{gu2016incorporating}.

On a boarder note,  more conferences on community detection are needed to offer researchers the opportunity to unite, exchange ideas and collaborate, .and make themselves known to researchers in other domains. Moreover, from a methodological perspective, deep learning should be a future trend in community detection, as graph neural networks (GNN) are already the main trend in graph mining. Future work can either transfer existing models such as modularity-based or statistical inference-based (stochastic block model) models to their deep learning version or develop new models via GNN to incorporate complex graphs ( i.e., large-scale graphs, dynamic graphs, and heterogeneous graphs). From an application perspective, better open-source frameworks need be developed and shared with researchers so that they can easily deploy and run models on those frameworks. Meanwhile, researchers should be attentive in exploring other scenarios where community detection techniques can be plugged in as extra support. 
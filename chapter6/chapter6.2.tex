\section{Limitation}

There are still some apparent limitations in the graph mining / community detection domain, the existing research environment, and the three projects introduced in this dissertation.

First, from a personal perspective, I firmly believe that graph mining needs be at the same level as natural language processing (NLP), computer vision (CV), and audio, as these four domains depict user profiles from four different perspectives: interaction, text, image, and audio. However, compared with the other three domains, graph mining suffers from a lack of attention In general. From an academic viewpoint, the other three domains have their particular top-tier journal and annual conference, such as ACL, CVPR, and ICASSP;  whereas for graph mining or network analysis, there are as yet no competitive conferences (NetSci is not a paper-driven conference). This reality leads to fewer chances for researchers in the complex network domain to exchange ideas and undertake collaborations. Meanwhile, at the university level, there are fewer institutes and research groups in graph mining, compared with the other three domains. Among the few groups, Indiana University Center for Complex Networks and Systems Research and Stanford University's Jure Leskovec group are the two most outstanding research units. From an industry viewpoint, there is still a long way to go before applying graph mining techniques in real-world scenarios. To date, most community detection research is still more theoretical than applicable, residing in the physics, statistics, and mathematics domains. Even though I can see more and more real applications involving graph mining and even community detection techniques, these techniques are used mostly as supportive features embedded in other application pipelines.

I propose the development of more independent applications solely use graph mining techniques. To my knowledge, Amazon and Microsoft have their own teams focusing on knowledge graphs such as Deep Graph Library\footnote{https://www.dgl.ai/}. Alibaba also releases its open-source deep learning based graph mining framework, Euler\footnote{https://github.com/alibaba/euler}, and the Graph Neural Network Platform, graph-learn\footnote{https://github.com/alibaba/graph-learn}. This is a good sign that high-tech companies are starting to pay more attention to the graph mining domain. 


Second, from a domain perspective, existing community detection researchers still face a lot of challenges. \cite{lancichinetti2011limits,xiang2012multi,xiang2012limitation} explains the resolution limit of modularity maximization.  \cite{kawamoto2015limitations} shows the limitations of spectral methods  with regard to detectability threshold and localization of eigenvectors. Because of the No Free Lunch Theorem \cite{wolpert1997no}, no single optimal method can satisfy all types of community detection tasks. To evaluate model performance, more reliable benchmark graphs and evaluation metrics need to be explored, proposed and constructed. Moreover, scalability is another issue. Many tasks have proved to be NP hard problems, which means they are not able to be solved in large-scale graphs. How to efficiently approximate optimal solutions and how to run possible solutions in parallel are the two main potential tracks to explore.

Third, from the dissertation perspective, in the literature review chapter, the selected works were filtered from a pool of over one thousand candidates. However, even though these works are the most representative ones, more than 70\% of high-quality works were filtered out. Moreover, the literature review is a brief summarization of several of the most popular topics and ignore other smaller topics such as multi-view graphs and hierarchical community detection. Therefore, although Chapter 2 gives considerable insight about the research that has occurred in the latest decade, not all topics are covered in detail.

There are also limitations with respect to each of the three research tasks presented here. In the first personalized community detection task,  the proposed model partially relies on existing models such as Infomap and Node2vec to construct the offline community tree and calculate pairwise node similarity. In the second cross-graph community detection task, In the cross-graph community detection task, it was necessary to empirically detect communities to generate pseudo-labels for training in a separate step,  which consumes time and lacks reliability. In the third community-aware product summarization task, community is leveraged as a strong support to generate product summarizations. In the proposed model, both product behavior representation and behavior-to-community pretraining need to be determined apart from the multi-task model. Besides, as the proposed task is quite new, there are no baseline models to compare it with. I have to compromise and apply several constructed step-by-step models instead of end-to-end models directly proposed for this task. 

\section{Contributions}
In this dissertation, I present an overview of community detection problems and three related research problems to explain how to solve community detection tasks and apply community information in other domains. Chapter 2 summarizes related works from various perspectives. Chapter 3 to Chapter 5 provide details on the three proposed research questions. In this Chapter, I  summarize the contribution of each chapter in my dissertation as follows.

Chapter 2 is the literature review discusses the most outstanding community detection researches, and summarizes more than twice the number of research studies noted in previously published literature. To my knowledge, Chapter 2 is the most comprehensive literature review in the community detection domain.  In this chapter, the selected papers were chosen from more than one thousand candidates are all published in the recent decade between 2010 and early 2020, which means they are timely, state-of-the-arts, and outstanding works. The selection criterion of paper was that a study must have been published in a top-tier journal or conference in its related domain or include at least 50 citations. I set 2020 as the time threshold to filter papers for two reasons: First, because several survey studies on community detection were published in the past five to ten years, I would duplicate my work if I paid too more attention to those classic works. Second, even though community detection is only a sub topic in graph mining, it has gained  exponentially increased attention in recent years. Many studies are published in computer science, physics, and statistics domains. It is necessary to summarize such works so that future researchers can have a better understanding and global view of the community detection domain. 

Based on ideas from previous researches \cite{fortunato2010community, fortunato2016community}, the literature reviewed in Chapter 2 includes five categories. First,  research to deal with different types of graphs is introduced. The graph types considered in this dissertation include heterogeneous and multilayer graphs, sparse graphs, dynamic graphs, large graphs and attribute graphs. The taxonomy covers most of the graph types that have appeared in community detection research. Second, different community detection tasks are also illustrated. As community detection is a complex domain, there are actually many works focusing on all kinds of trivial and subtle tasks. Among these numerous tasks, I selected five of the most interesting tasks and introduced ten to twenty works for each task: overlapping community detection task aims to solve the typical scenario where a node can belong to multiple communities; number of communities task argues how to explore and choose the best number of communities for many existing approaches; community search task is a type of partial community detection mixed with information retrieval; and enhanced and semi-supervised community detection are  two other tasks widely discussed in current research studies. I also discuss how to apply community detection to other domains and how to evaluate and compare model performances in the dissertation. In general, these five mentioned categories cover most topics in community detection, which I trust I presented with clarity and preciseness in an easy-to-follow way.

In Chapter 3, I introduce a personalized community detection approach, which is the first attempt to address the topic of detecting communities with different resolutions in an attempt to meet user need. To solve this task, I proposed a model with an offline binary community tree construction step and an online genetic pruning step. A distributed version of the model is also deployed to accelerate running efficiency. Extensive experiments on two different datasets show that my proposed model outperforms all baselines in terms of accuracy and efficiency. 

In Chapter 4, I address an outgrowth of the star-shaped topology of the current internet ecosystem. For example, giant service providers, such as Facebook, Google, and Amazon provide easy login channels to thousands of sites and apps, which means that users no longer need to undergo laborious account creation processes. This situation inspired me to propose a novel cross-graph community detection inquiry in order to detect user communities in sparse graphs by leveraging cross-graph information and pairwise learning techniques. This knowledge can substantially help small businesses and services identify and understand user groups and their interests. In the proposed model, a two-level filtering module helps to reduce the negative propagation effect from noisy and heterogeneous graphs. A well-defined Community Recurrent Unit (CRU) detects user community affiliations to support pairwise community closeness prediction. Extensive experiments on two real datasets demonstrate the superiority of the model compared with all baselines.

In Chapter 5, by leveraging communities detected from multi-type user behaviors instead of sparse reviews, I propose a multi-task model to solve an innovative dynamic product summarization task. Extensive experiments show the model is consistently promising and significantly outperforms the baselines.  Being the first study on this newly proposed task, I aim to explore the relationship between user community and product summarization so as to address the cold-start problem, that it, products without recent reviews. As a result, the proposed model can provide a much larger scope in the e-commerce ecosystem while enabling explainable dynamic analysis on products.   As the generated summarization is sensitive to customers, I would never wish to generate ‘fake reviews’ to mislead them. Instead, the summarization should be provided to online sellers only as auxiliary information.

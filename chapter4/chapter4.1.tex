\section{Introduction}

Community detection is an essential task for cyberspace mining, which has been successfully employed to explore users’ resemblance for retrieval/recommendation enhancement and user behavior analysis. Taking social media and e-commerce as examples, the complex, and often heterogeneous, relations among users and other objects, e.g., products, reviews, and messages, can be encapsulated as bipartite graphs, and the topology can help to synthesize and represent users with a coarser and broader view.

While a graph is well-connected, conventional methods, e.g., modularity-based approach \cite{newman2004fast}, spectral approach \cite{nascimento2011spectral}, dynamic approach \cite{peixoto2017modelling} and deep learning approach \cite{chiang2019cluster}, are able to estimate the internal/external connectivity and generate high-quality communities directly on nodes \cite{fortunato2016community}.
 
For sparse graphs, as mentioned in the Chapter \ref{ch:intro}, to deal with the lack-of-connection problem, I propose  a novel research task – Cross-Graph Community Detection. The idea is based on the fact that an increasing number of small apps are utilizing the user identity information inherited from giant providers, i.e., users can easily login a large number of new apps by using Facebook and Google ID. In such ecosystem, the main large graph can provide critical information to enlighten the community detection on many small sparse graphs. Note that, in spit of the small sparse graphs can engage with a specific field, the main graph is quite comprehensive and noisy. As Figure \ref{fig:c4_example} shows, not all the connections in Amazon (shopping graph) can be equally important for the two candidate app graphs. In the example, three mutual users are selected where $u_1$ and $u_2$ mainly share similar shopping interests on cosmetics and $u_1$ and $u_3$ mainly share similar shopping interests on food products in Amazon. Then, with deliberate propagation from main graph, in the Cosmetic graph, $u_1$ and $u_2$ have a better chance to be grouped together, while $u_1$ and $u_3$ are more likely to be assigned the same community ID in the Cooking graph. Therefore, the proposed model should be able to differentiate various kinds of information from the main graph for each candidate sparse graph to enhance its local community detection performance. 

As another challenge, small sparse graphs often suffer from training data insufficiency, e.g., the limited connections in these graphs can hardly tell the community residency information. In this study, I employed a novel data augmentation approach - cross-graph pairwise learning. Given a candidate user and an associated user triplet, the proposed model can detection the community closeness superiority by leveraging main graph and the sparse graph simultaneously. Moreover, the proposed pairwise learning method can cope with the main graph heterogeneity issue and reduce noisy information by taking care of graph local structure. Theoretically, I can offer at most $\mathcal{O}(N^{3})$ user triplets to learn graph community structure while conventional community detection methods by default can only be applied on $\mathcal{O}(N)$ users  ($N$ is the number of users in the sparse graph).

Therefore, I propose an innovative \textit{Pairwise Cross-graph Community Detection} (PCCD) model for enhanced sparse graph user community detection. Specifically, given user $u_i$ and its associated triplet $\langle u_{i},u_{j},u_{k}\rangle$, I aim to predict their pairwise community relationship, e.g., compared with user $u_{k}$, user $u_{j}$ should have closer, similar or farther community closeness to user $u_i$. 

%%%%%%%%%%%%%%%%%%%%%%%%%%%%%%%%%%%%%%%%%%%%%%%%%%%%%%%%%
% Do not edit these lines unless you wish to customize
% the template
%%%%%%%%%%%%%%%%%%%%%%%%%%%%%%%%%%%%%%%%%%%%%%%%%%%%%%%%%
%\newgeometry{left=1in}

\begin{center}

\yourName\\
\thesisTitle

\end{center}
\vspace{1.5\baselineskip}

Community detection has always been one of the fundamental research questions in graph mining. As a type of unsupervised or semi-supervised approach, community detection aims to explore graphs of hidden patterns merely relying on graph topological structure. By grouping similar nodes or edges into the same community while separate dissimilar ones apart from different communities, graph structure can be revealed in a coarser resolution. It can be beneficial for numerous applications such as user shopping recommendation and advertisement in e-commerce, protein-protein interaction prediction in the biomedical domain,  and literature recommendation or scholar collaboration in citation analysis.

However, identifying communities is an ill-defined problem. Due to the No Free Lunch theorem, there is neither gold standard to represent perfect community partition nor universal methods that are able to detect satisfied communities for all community detection tasks in various types of graphs. To have a global view of this research question, I summarize state-of-art community detection methods by categorizing them based on graph types, research tasks and utilized method tracks. As academic researches for community detection grows rapidly in recent years, I hereby particularly focus on the works published in the recent decade, which may leave out some classic models published decades ago. 

Meanwhile, three subtle community detection tasks are proposed and assessed in this dissertation as well. First, apart from general models which only consider graph structure for community partition, personalized community detection considers user need as auxiliary information. In the end, there will be higher resolution communities for nodes better matching user needs while coarser-resolution communities for those less relevant nodes. Second, graphs always suffer from sparse connectivity issues. Leveraging conventional models directly on such graphs may hugely distort the quality of generating communities. To tackle such a problem, cross-graph techniques are involved to propagate external graph information as a support for current graph community detection. Third, as a type of unsupervised or semi-supervised learning method,  community partition results are lack of explanations. The final output is only nodes/edges clusters with no further in-depth interpretation. By involving natural language processing techniques, a joint learning model is applied to not only detect node communities but also generate a short description for node intrinsic characteristics.  

The contribution of this dissertation is threefold. First, a decent amount of recent works are summarized under a well-defined taxonomy. Existing works about methods, evaluation and applications are all addressed in proper order. Second, three novel community detection tasks are proposed and their associated models are evaluated by comparing with alternative baselines under various datasets. Third,  the limitations of current works and several future tracks with potential academic and industrial value are also discussed in the final chapter. 


\ifdefined\committeeMemberFourTypedName

\null\hfill \myRule\\
%\null\hfill \committeeChairpersonTypedName, \committeeChairpersonPostNominalInitials\\
\null\hfill Xiaozhong Liu, Ph.D. (Principle Advisor)\\ 
\null\hfill \\
\null\hfill \myRule\\
\null\hfill Xiaofeng Wang, Ph.D.\\
\null\hfill \\
\null\hfill \myRule\\
\null\hfill Yong-Yeol Ahn, Ph.D.\\
\null\hfill \\
\null\hfill \myRule\\
\null\hfill Kahyun Choi, Ph.D.\\


\ifdefined\committeeMemberFiveTypedName
\null\hfill \myRule\\
\null\hfill \committeeMemberFiveTypedName, \committeeMemberFivePostNominalInitials\\
\fi

\fi
\restoregeometry

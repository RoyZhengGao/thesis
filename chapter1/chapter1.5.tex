\section{Thesis Structure}
The remaining chapters of the thesis are organized as follows. In Chapter \ref{ch:review}, it iss a literature review of the most representative papers published in the recent decade. Selected out of over one thousand papers, the studies mentioned cover the most popular topics in community detection following a well-defined taxonomy. I summarize papers from five different perspectives, including graph types, different tasks, main track methods, applications and evaluation. Within each section, I further organize relevant papers into different sub-categories. For example, the graph type section introduces papers dealing with community detection problems on heterogeneous \& multi-layer graphs, sparse graphs, dynamic graphs, large graphs, and attribute graphs. In this hierarchical taxonomy, papers are best distinguished and separated into different topics.The Chapter \ref{ch:personalized}  introduces the first proposed task: personalized community detection. I introduce the background and motivation of the research question, propose my own model, run experiments to compare with other state-of-the-art baselines, and discuss experimental results to infer some in-depth understandings. Chapter \ref{ch:cross-graph} and Chapter \ref{ch:community-aware} follow a similar structure to introduce the other two proposed tasks, namely, cross-graph community detection and community-aware dynamic product summarization. Finally, in Chapter \ref{ch:conclusion}, I summarize the overall contribution of this dissertation in detail, point out current limitations in both this dissertation and the community detection domain as a whole, and suggest some ways of overcoming these limitations through future research.


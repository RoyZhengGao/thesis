\section{Background and Motivation}
Graph mining is one of the fundamental research areas in artificial intelligence; in the recent decade, community detection continues to be an essential topic within graph mining. Community detection approaches group nodes into communities, which can be regarded as a coarser view of a graph and reveal the latent knowledge of nodes. This detected knowledge can be used to support further tasks such as link prediction or node profile construction. For example, detecting researcher communities in a scholarly graph can imply future collaboration between them, or offer paper/research recommendations to targeted users. In biology, community structure can be used to infer possible protein-protein binding relationships, which may potentially contribute to healthcare. Therefore, exploring community detection tasks has huge practical value in both academic and industrial scenarios.

The popularity of community detection research is suggested by the thousands of papers published on the subject in top-tier journals and presented at conferences. Among these, some are survey papers which summarize state-of-the-art models from different perspectives; these have served as great references for me to organize relevant papers into different categories. \cite{fortunato2010community, fortunato2016community, coscia2011classification} are general survey papers which broadly introduce the definition of community detection and several types of models. Other papers pay more attention to detecting communities in particular types of graph. For example, \cite{malliaros2013clustering} is a classic paper which summarizes community detection in directed graphs; \cite{harenberg2014community} particularly focuses on large-scale graphs with empirical studies; and \cite{kim2015community} particularly summarizes the latest community detection models for multi-layer graphs. Still other papers focus on particular types of model frameworks. For example, \cite{abbe2017community} is the most recent work on block models in community detection. \cite{xie2013overlapping,amelio2014overlapping} are both authentic survey papers on overlapping community detection, even though they are slightly dated as of 2020.  \cite{nascimento2011spectral} is a classic research work that introduces a collection of spectral clustering models. To demonstrate how community detection supports interdisciplinary research efforts, \cite{javed2018community} and \cite{bedi2016community} show the application of community models to social media mining as well as other domains such as biology and neuroscience. Finally, \cite{chakraborty2017metrics} summarizes all types of evaluation metrics for model performance comparison. 

However, all of the aforementioned papers are either slightly out of date or relevant only to a specific topic; what is lacking is an effort to summarize and track the latest works from all possible main perspectives. This inspired me to include the most representative papers in this thesis, categorized by my own defined taxonomy. Specifically, derived from all existing works, I aim to leverage in-depth analysis in some subtle tasks and propose novel models to improve current model performance. A survey of the existing literature also inspired me to formulate and solve the various research questions mentioned in the following section.
\chapter{Introduction}
focus on community detection. introduce the overall development in the recent decade and particularly address three subtle tasks.


summary or empirical studies.
graph types: heterogeneous, directed, weighted, sparse, temporal(time involving), hyper graph,multi-layer,large scale
different tasks: math proof (threshold), overlapping, number of communities, top-k community search// subgraph, Enhancing community detection, semi-supervised community detection
applications: biology, citation analysis, transportation, social network
methods: Modularity, Spectral, Stochastic block model,  deep learning, nmf,Multiscale (hierarchical), multi-view, with enriched information, community recovery, flow based(random walk), link based, motif, others
challenges: resolution limit, unbalanced groups,
evaluation, software, datasets


so far community detection is more like a statistic/ physics questions instead of computer science and arificial intelligence question. 
modularity -> sbm -> deep learning. modularity and sbm are more clear tasks. deep learning methods are fuzzy.
spectral clustering all the time


future tracks: from computer science perspective, deep learning, random graph? hyper graph? multiscale?

findings: biology has delay, not easy to get a lot citations in short time.

concept level difference between subgraph, community, motif, cliques: community is from node side, others are from graph side. it is a part of community search.

overlapping has huge correlation with nmf
spetral clustering and sbm are highly correlated

some tracks are huge: modularity (newman), stochastic block model (newman) and overlapping (Jure)

try my best to distinguish each category, but there are still some overlaps between them. Some algorithms are both for overlapping community detection using deep learning techniques.

mention other types of approaches a bit, such as link community detection, 

summarize some trend, well known professor/teams, by years. which venue tends to be more popular.

I manually code five hundred most influential papers in community detection domain.
\cite{newman2012communities}
\section{Motivation}
why I do community detection?
Network Science, which is also called graph mining or complex network analysis, is one of the fundamental research questions in artificial intelligence. 

\section{Background and Motivation}
Graph mining is one of the fundamental research areas in artificial intelligence; in the recent decade, community detection continues to be an essential topic within graph mining. Community detection approaches group nodes into communities, which can be regarded as a coarser view of a graph and reveal the latent knowledge of nodes. This detected knowledge can be used to support further tasks such as link prediction or node profile construction. For example, detecting researcher communities in a scholarly graph can imply future collaboration between them, or offer paper/research recommendations to targeted users. In biology, community structure can be used to infer possible protein-protein binding relationships, which may potentially contribute to healthcare. Therefore, exploring community detection tasks has huge practical value in both academic and industrial scenarios.

The popularity of community detection research is suggested by the thousands of papers published on the subject in top-tier journals and presented at conferences. Among these, some are survey papers which summarize state-of-the-art models from different perspectives; these have served as great references for me to organize relevant papers into different categories. \cite{fortunato2010community, fortunato2016community, coscia2011classification} are general survey papers which broadly introduce the definition of community detection and several types of models. Other papers pay more attention to detecting communities in particular types of graph. For example, \cite{malliaros2013clustering} is a classic paper which summarizes community detection in directed graphs; \cite{harenberg2014community} particularly focuses on large-scale graphs with empirical studies; and \cite{kim2015community} particularly summarizes the latest community detection models for multi-layer graphs. Still other papers focus on particular types of model frameworks. For example, \cite{abbe2017community} is the most recent work on block models in community detection. \cite{xie2013overlapping,amelio2014overlapping} are both authentic survey papers on overlapping community detection, even though they are slightly dated as of 2020.  \cite{nascimento2011spectral} is a classic research work that introduces a collection of spectral clustering models. To demonstrate how community detection supports interdisciplinary research efforts, \cite{javed2018community} and \cite{bedi2016community} show the application of community models to social media mining as well as other domains such as biology and neuroscience. Finally, \cite{chakraborty2017metrics} summarizes all types of evaluation metrics for model performance comparison. 

However, all of the aforementioned papers are either slightly out of date or relevant only to a specific topic; what is lacking is an effort to summarize and track the latest works from all possible main perspectives. This inspired me to include the most representative papers in this thesis, categorized by my own defined taxonomy. Specifically, derived from all existing works, I aim to leverage in-depth analysis in some subtle tasks and propose novel models to improve current model performance. A survey of the existing literature also inspired me to formulate and solve the various research questions mentioned in the following section.
\section{Research Questions}
raise the three sub topics mentioned in the later experiements.
\section{Contribution}
explain the contribution of this thesis such as categorizing the community detection into several forms, solve the three research questions, etc.
\section{Thesis Structure}
The remaining chapters of the thesis are organized as follows. In Chapter \ref{ch:review}, it iss a literature review of the most representative papers published in the recent decade. Selected out of over one thousand papers, the studies mentioned cover the most popular topics in community detection following a well-defined taxonomy. I summarize papers from five different perspectives, including graph types, different tasks, main track methods, applications and evaluation. Within each section, I further organize relevant papers into different sub-categories. For example, the graph type section introduces papers dealing with community detection problems on heterogeneous \& multi-layer graphs, sparse graphs, dynamic graphs, large graphs, and attribute graphs. In this hierarchical taxonomy, papers are best distinguished and separated into different topics.The Chapter \ref{ch:personalized}  introduces the first proposed task: personalized community detection. I introduce the background and motivation of the research question, propose my own model, run experiments to compare with other state-of-the-art baselines, and discuss experimental results to infer some in-depth understandings. Chapter \ref{ch:cross-graph} and Chapter \ref{ch:community-aware} follow a similar structure to introduce the other two proposed tasks, namely, cross-graph community detection and community-aware dynamic product summarization. Finally, in Chapter \ref{ch:conclusion}, I summarize the overall contribution of this dissertation in detail, point out current limitations in both this dissertation and the community detection domain as a whole, and suggest some ways of overcoming these limitations through future research.




\section{Introduction}
Community detection is an important topic in graph mining. By learning node community labels on the graph, we are able to detect node hidden attributes as well as explore the closeness between nodes \cite{fortunato2010community}. Conventional methods are mostly user-independent to detect communities solely relying on graph topological structure \cite{fortunato2016community}, generate semi-supervised communities with node constraints \cite{jin2019graph}, or select top-K sub graphs as user-centric communities \cite{li2015influential}. These approaches are no longer enough to satisfy users with a pursuit of personalization, which makes involving user need into community detection to become an inevitable task.

First, user-independent approaches solely consider graph topological structure without user need. Second, semi-supervised approaches detect communities restricted by pre-selected seed nodes. As different user needs refer to different seeds, each individual user requires a separate process to run the whole model completely to get personalized communities, which is inapplicable in real cases. Third, sub-graph selection approaches only generate communities from the partial graph instead of the whole one.  

To detect personalized communities on the whole graph, I propose a \textbf{g}enetic \textbf{P}ersonalized \textbf{C}ommunity \textbf{D}etection (gPCD) model with an offline and an online step. Specifically, in the offline step, I convert the user-independent graph community to a binary community tree which is encoded with binary code. Subsequently, a deep learning method is utilized to learn low-dimensional embedding representations for both user need and nodes on the binary community tree. In the online step, I propose a genetic tree-pruning approach on the tree to detect personalized communities by maximizing user need and minimizing user searching cost simultaneously. The whole genetic approach runs in an iterative manner to simulate an evolutionary process and generate a number of partition candidates which are regarded as ``chromosomes'' in each genetic generation. Through the selection, cross-over and mutation process, successive chromosomes are bred as better personalized community partitions to meet with user need.
